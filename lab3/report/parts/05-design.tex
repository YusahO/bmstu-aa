\chapter{Конструкторская часть}

В данном разделе будут приведены схемы алгоритмов умножения матриц, описание используемых типов данных и структуры программного обеспечения.

\section{Требования к ПО}

К программе предъявлен ряд требований:

\begin{itemize}
    \item на вход программе подаются две матрицы, каждая записана в отдельном текстовом файле;
    \item результатом умножения является матрица, выводимая на экран;
    \item программа должна позволять производить измерения процессорного времени, затрачиваемого на выполнение реализуемых алгоритмов.
\end{itemize}

\section{Разработка алгоритмов}

\section{Описание используемых типов данных}

При реализации алгоритмов будут использованы следующие типы данных:

\begin{itemize}
    \item \textit{матрица}~--- двумерный массив значений типа \texttt{int}.
\end{itemize}

\section{Модель вычисления для проведения оценки трудоемкости}

Введем модель вычислений, которая потребуется для определения трудоемкости каждого отдельного взятого алгоритма умножения матриц.

\begin{enumerate}
    \item Трудоемкость базовых операций имеет:
    \begin{itemize}
        \item значение 1 для операций:
        \begin{equation}
            \begin{gathered}
                +, -, =, +=, -=, ==, !=, <, >, <=, >=, \\ 
                \text{[]}, ++, --, \&\&, ||, >>, <<, \&, |
            \end{gathered}
        \end{equation}
        \item значение 2 для операций:
        \begin{equation}
            *, /, \%, *=, /=, \%=.
        \end{equation}
    \end{itemize}
    \item Трудоемкость условного оператора:
    \begin{equation}
        f_{\text{if}} =
        \begin{cases}
            \min({f_1, f_2}), & \text{лучший случай} \\
            \max({f_1, f_2}), & \text{худший случай}.
        \end{cases}
    \end{equation}
    \item Трудоемкость цикла
    \begin{equation}
        \begin{gathered}
            f_{\text{for}} = f_{\text{инициализация}} + f_{\text{сравнение}} + \\
            + M_{\text{итераций}} \cdot (f_{\text{тело}} + f_{\text{инкремент}} + f_{\text{сравнение}}).
        \end{gathered} 
    \end{equation}
    \item Трудоемкость передачи параметра в функцию и возврат из нее равен 0.
\end{enumerate}

\section{Трудоемкость алгоритмов}

\section*{Вывод}
\addcontentsline{toc}{section}{Вывод}