\chapter{Конструкторская часть}

В данном разделе будут приведены псевдокоды алгоритмов блинной, быстрой и гномьей сортировок, оценки их трудоемкостей.

\section{Требования к программному обеспечению}

К программе предъявлен ряд требований:
\begin{itemize}
    \item на вход программе подается два массива;
    \item результатом сортировки является массив, выводимый на экран;
    \item программа должна позволять производить измерения процессорного времени, затрачиваемого на выполнение реализуемых алгоритмов.
\end{itemize}

\section{Разработка алгоритмов}

На листинге \ref{alg:pancake-sort} приведен псевдокод алгоритма блинной сортировки, на \ref{alg:pancake-sort-flip}~--- псевдокод вспомогательной функции \texttt{Flip}.

На листинге \ref{alg:q-sort} приведен псевдокод алгоритма быстрой сортировки (разбиение Хоара).

На листинге \ref{alg:gnome-sort} приведен псевдокод алгоритма гномьей сортировки.

Приведенные реализации предполагают, что на вход подаются корректные данные.

\begin{algorithm}[H]
    \caption{Вспомогательная функция Flip}
    \label{alg:pancake-sort-flip}
    \begin{algorithmic}[1]
        \Procedure{Flip}{$arr$, $maxNumPos$, $n$}
        \State $i\gets maxNumPos$
        \While{$i < n$}
        \State $swap\gets arr[i]$
        \State $arr[i]\gets arr[n]$
        \State $arr[n]\gets swap$
        \State $n\gets n - 1$
        \State $i\gets i + 1$
        \EndWhile
        \EndProcedure
    \end{algorithmic}
\end{algorithm}

\begin{algorithm}[H]
    \caption{Алгоритм блинной сортировки}
    \label{alg:pancake-sort}
    \text{\textbf{Вход}: ссылка на массив $arr$ размером $n$}\\
    \text{\textbf{Выход}: отсортированный массив $arr$}

    \begin{algorithmic}[1]
        \Procedure{PancakeSort}{$arr$, $n$}
        \If{$n \ge 2$}
        \State $i\gets n$
        \While{$i > 1$}
        \State $maxNumPos\gets 0$
        \For{$a\gets 0, i$}
        \If{$arr[i] > arr[maxNumPos]$}
        \State $maxNumPos\gets a$
        \EndIf
        \EndFor
        \If{$maxNumPos\ne (i - 1)$ \textbf{and} $maxNumPos\ge 0$}
        \State \Call{Flip}{$arr, maxNumPos, i$}
        \EndIf
        \EndWhile
        \EndIf
        \EndProcedure
    \end{algorithmic}
\end{algorithm}

\begin{algorithm}[H]
    \caption{Алгоритм быстрой сортировки (разбиение Хоара)}
    \label{alg:q-sort}
    \text{\textbf{Вход}: ссылка на массив $arr$, индексы $st$ и $end$}\\
    \text{\textbf{Выход}: отсортированный массив $arr$}

    \begin{algorithmic}[1]
        \Procedure{QSortHoare}{$arr$, $st$, $end$}
        \If{$st\ge 0$ \textbf{and} $end\ge 0$ \textbf{and} $st < end$}
            \State $l\gets st$
            \State $r\gets end$
            \State $piv\gets arr[\frac{l + r}{2}]$
            \Loop
                \Do
                    \State $l\gets l + 1$
                \doWhile{$arr[l] < piv$}
                \Do
                    \State $r\gets r - 1$
                \doWhile{$arr[r] > piv$}
                \If{$l < r$}
                    \State \textbf{break}
                \EndIf
                \State $swap\gets arr[l]$
                \State $arr[l]\gets arr[r]$
                \State $arr[r]\gets swap$
            \EndLoop
            \State \Call{QSortHoare}{$arr, st, r$}
            \State \Call{QSortHoare}{$arr, r + 1, end$}
        \EndIf
        \EndProcedure
    \end{algorithmic}
\end{algorithm}

\begin{algorithm}[H]
    \caption{Алгоритм гномьей сортировки}
    \label{alg:gnome-sort}
    \text{\textbf{Вход}: ссылка на массив $arr$ размером $n$}\\
    \text{\textbf{Выход}: отсортированный массив $arr$}

    \begin{algorithmic}[1]
        \Procedure{GnomeSort}{$arr$, $n$}
        \State $i\gets 1$
        \State $j\gets 2$
        \While{$i < n$}
            \If{$arr[i - 1] < arr[i]$}
                \State $i\gets j$
                \State $j\gets j + 1$
            \Else
                \State $swap\gets arr[i - 1]$
                \State $arr[i - 1]\gets arr[i]$
                \State $arr[i]\gets swap$
                \State $i\gets i - 1$
                \If{$i == 0$}
                    \State $i\gets j$
                    \State $j\gets j + 1$
                \EndIf
            \EndIf
        \EndWhile
        \EndProcedure
    \end{algorithmic}
\end{algorithm}

\section{Описание используемых типов данных}

При реализации алгоритмов будут использованы следующие типы данных:
\begin{itemize}
    \item \textit{массив}~--- динамический массив значений целого типа.
\end{itemize}

\section{Модель вычисления для проведения оценки трудоемкости}

Введена модель вычислений, которая потребуется для определения трудоемкости каждого отдельного взятого алгоритма умножения матриц.
\begin{enumerate}
    \item Трудоемкость базовых операций имеет:
    \begin{itemize}
        \item значение 1 для операций:
        \begin{equation}
            \begin{gathered}
                +, -, =, +=, -=, ==, !=, <, >, <=, >=, \\ 
                \text{[]}, ++, --, \&\&, ||, >>, <<, \&, |
            \end{gathered}
        \end{equation}
        \item значение 2 для операций:
        \begin{equation}
            *, /, \%, *=, /=, \%=.
        \end{equation}
    \end{itemize}
    \item Трудоемкость условного оператора:
    \begin{equation}
        f_{\text{if}} =
        \begin{cases}
            \min({f_1, f_2}), & \text{лучший случай} \\
            \max({f_1, f_2}), & \text{худший случай}.
        \end{cases}
    \end{equation}
    \item Трудоемкость цикла
    \begin{equation}
        \begin{gathered}
            f_{\text{for}} = f_{\text{инициализация}} + f_{\text{сравнение}} + \\
            + M_{\text{итераций}} \cdot (f_{\text{тело}} + f_{\text{инкремент}} + f_{\text{сравнение}}).
        \end{gathered} 
    \end{equation}
    \item Трудоемкость передачи параметра в функцию и возврат из нее равен 0.
\end{enumerate}

\section{Трудоемкость алгоритмов}

Ниже приведены оценки трудоемкостей алгоритмов блинной, быстрой и гномьей сортировок.

Пусть требуется отсортировать массив $A$ размером $N$.

\section{Блинная сортировка}

Трудоемкость блинной сортировки складывается из:
\begin{itemize}
    \item трудоемкости прохода по $N - 1$ элементам массива
    \begin{equation}
        f_{mainPass} = 2 + (N - 1) \cdot (2 + f_{body} + f_{flip})
    \end{equation}
    \item трудоемкости прохода по $N - 1$ элементам массива во внутреннем цикле
    \begin{equation}
        f_{body} = 2 + (N - 1) \cdot \left (2 + 3 + 3 +
        \begin{cases}
            0, & \text{лучший случай} \\
            1, & \text{худший случай}
        \end{cases} \right )
    \end{equation}
    \item трудоемкости переворачивания элементов массива
    \begin{equation}
        f_{flip} = 3 + 
        \begin{cases}
            0, & \text{лучший случай} \\
            2 + (N - 2) \cdot (2 + 7), & \text{худший случай}
        \end{cases}
    \end{equation}
\end{itemize}

Таким образом, трудоемкость блинной сортировки в худшем случае равна
\begin{equation}
    \begin{gathered}
        f_{PanSortWorst} = 2 + (N - 1) \cdot (4 + 9(N - 1) + 5 + 9(N - 2)) = \\
        = 18N^2 - 36N + 20 \approx O(N^2)
    \end{gathered}
\end{equation}
и в лучшем случае:
\begin{equation}
        f_{PanSortBest} = 2 + (N - 1) \cdot (4 + 8(N - 1) + 3) = 8N^2 - 9N + 3 \approx O(N^2)
\end{equation}

\section{Быстрая сортировка}

Трудоемкость быстрой сортировки (с выбором среднего элемента в качестве опорного) складывается из:
\begin{itemize}
    \item трудоемкости прохода по $N$ элементам массива (или подмассива)
    \begin{equation}
        f_{body} = f_{split} \cdot (5 + 7 + N \cdot (4 + f_{swap}))
    \end{equation}
    \item трудоемкости перестановки элементов
    \begin{equation}
        f_{swap} = 1 + 
        \begin{cases}
            0, & \text{лучший случай}\\
            2 + 3 + 2 + 2, & \text{худший случай}
        \end{cases}
    \end{equation}
    \item трудоемкости разбиения массива
    \begin{equation}
        f_{split} = 
        \begin{cases}
            \log_2{N}, & \text{лучший случай} \\
            N, & \text{худший случай}
        \end{cases}
    \end{equation}
\end{itemize}

Таким образом, трудоемкость быстрой сортировки в худшем случае равна
\begin{equation}
    f_{QSortWorst} = N \cdot (12 + 14N) = 12N + 14N^2 \approx O(N^2)
\end{equation}
и в лучшем случае:
\begin{equation}
    f_{QSortBest} = \log_2{N} (12 + 5N) = 12\log_2{N} + 5N\log_2{N} \approx O(N\log_2{N})
\end{equation}

\section{Гномья сортировка}

Трудоемкость гномьей сортировки складывается из:
\begin{itemize}
    \item трудоемкости прохода по $N - 1$ элементам массива
    \begin{equation}
        f_{body} = 2 + 2 + (N - 1) \cdot (2 + f_{backward})
    \end{equation}
    \item трудоемкости обратного прохода по $N$ элементам массива для выполнения их обмена
    \begin{equation}
        f_{backward} = 6 +
        \begin{cases}
            1, & \text{лучший случай}\\
            N \cdot (3 + 4 + 2 + 1), & \text{худший случай}
        \end{cases}
    \end{equation}
\end{itemize}

Таким образом, трудоемкость гномьей сортировки в худшем случае равна
\begin{equation}
    \begin{gathered}
        f_{GnSortWorst} = 4 + (N - 1) \cdot (8 + 10N) = 10N^2 - 2N - 4 \approx O(N^2)
    \end{gathered}
\end{equation}
и в лучшем случае:
\begin{equation}
    f_{GnSortBest} = 4 + (N - 1) \cdot (8 + 1) = 9N - 5 \approx O(N)
\end{equation}

\section*{Вывод}
\addcontentsline{toc}{section}{Вывод}

В данном разделе на основе теоретических данных были построены схемы реализуемых алгоритмов сортировки.
Также была проведена оценка трудоемкостей алгоритмов.