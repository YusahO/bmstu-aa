\chapter{Технологическая часть}

В данном разделе приведены средства реализации программного обеспечения, сведения о модулях программы, листинг кода и функциональные тесты.

\section{Средства реализации}

В качестве языка программирования, используемого при написании данной лабораторной работы, был выбран C++ \cite{cpp-lang}, так как в нем имеется контейнер \texttt{std::vector}, представляющий собой динамический массив данных произвольного типа, и библиотека \texttt{<ctime>} \cite{ctime}, позволяющая производить замеры процессорного времени.

В качестве средства написания кода была выбрана кроссплатформенная среда разработки \textit{Visual Studio Code} за счет того, что она предоставляет функционал для проектирования, разработки и отладки ПО.

\section{Сведения о модулях программы}

Данная программа разбита на следующие модули:

\begin{itemize}
    \item \texttt{main.cpp}~--- файл, содержащий точку входа в программу;
    \item \texttt{algorithms.cpp}~--- файл, содержащий реализации алгоритмов сортировки;
    \item \texttt{measure.cpp}~--- файл, содержащий функции, замеряющие процессорное время выполнения реализуемых алгоритмов.
\end{itemize}

\clearpage
\section{Реализация алгоритмов}

\section*{Вывод}
\addcontentsline{toc}{section}{Вывод}

Были реализованы алгоритмы сортировки (блинная, быстрая, гномья).
Проведено тестирование реализованных алгортимов.
