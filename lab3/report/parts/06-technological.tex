\chapter{Технологическая часть}

В данном разделе приведены средства реализации программного обеспечения, сведения о модулях программы, листинг кода и функциональные тесты.

\section{Средства реализации}

В качестве языка программирования, используемого при написании данной лабораторной работы, был выбран C++ \cite{cpp-lang}, так как в нем имеется контейнер \texttt{std::vector}, представляющий собой динамический массив данных произвольного типа, и библиотека \texttt{<ctime>} \cite{ctime}, позволяющая производить замеры процессорного времени.

В качестве средства написания кода была выбрана кроссплатформенная среда разработки \textit{Visual Studio Code} за счет того, что она предоставляет функционал для проектирования, разработки и отладки ПО.

\section{Сведения о модулях программы}

Данная программа разбита на следующие модули:

\begin{itemize}
    \item \texttt{main.cpp}~--- файл, содержащий точку входа в программу;
    \item \texttt{algorithms.cpp}~--- файл, содержащий реализации алгоритмов сортировки;
    \item \texttt{measure.cpp}~--- файл, содержащий функции, замеряющие процессорное время выполнения реализуемых алгоритмов.
\end{itemize}

\clearpage
\section{Реализация алгоритмов}

На листинге \ref{lst:pancake-sort} представлена реализация алгоритма блинной сортировки.

\begin{lstlisting}[caption={Реализация алгоритма блинной сортировки и вспомогательной функции \texttt{flip}}, label={lst:pancake-sort}]
static void flip(std::vector<int> &arr, int m, int n)
{
    int swap, i;
    for (i = m; i < --n; i++)
    {
        swap = arr[i];
        arr[i] = arr[n];
        arr[n] = swap;
    }
}
void PancakeSort(std::vector<int> &arr)
{
    if (arr.size() < 2)
        return;
    for (int i = arr.size(); i > 1; --i)
    {
        int maxNumPos = 0;
        for (int a = 0; a < i; ++a)
            if (arr[a] > arr[maxNumPos])
                maxNumPos = a;

        if (maxNumPos == (i - 1))
            continue;
        if (maxNumPos >= 0)
            flip(arr, maxNumPos, i);
    }
}
\end{lstlisting}

\clearpage
\begin{lstlisting}[caption={Реализация алгоритма гномьей сортировки}, label={lst:gnome-sort}]
void GnomeSort(std::vector<int> &arr)
{
    int i = 1, j = 2;
    while (i < (int)arr.size())
    {
        if (arr[i - 1] < arr[i])
        {
            i = j++;
        }
        else
        {
            int swap = arr[i - 1];
            arr[i - 1] = arr[i];
            arr[i] = swap;
            --i;
            if (i == 0)
                i = j++;
        }
    }
}
\end{lstlisting}   

\begin{lstlisting}[caption={Реализация алгоритма быстрой сортировки}, label={lst:q-sort}]
void QuickSort(std::vector<int> &arr, int start, int end)
{
    if (start < 0 || end < 0 || start >= end)
        return;

    int pivot = arr[start];
    int l = start - 1, r = end + 1;
    while (true)
    {
        do
        {
            ++l;
        } while (arr[l] < pivot);
        do
        {
            --r;
        } while (arr[r] > pivot);

        if (l >= r)
            break;

        int swap = arr[l];
        arr[l] = arr[r];
        arr[r] = swap;
    }
    QuickSort(arr, start, r);
    QuickSort(arr, r + 1, end);
}
\end{lstlisting}


\section{Функциональные тесты}

На таблице \ref{tbl:func-tests} представлены функциональные тесты алгоритмов сортировки.

\begin{table}[H]
    \small
	\begin{center}
		\caption{Функциональные тесты алгоритмов сортировки}
		\label{tbl:func-tests}  
            \begin{tabular}{|c|c|c|c|c|}
            \hline
            & \multicolumn{4}{c|}{\textbf{Результат}} \\
            \cline{2-5}
            \textbf{Массив} & \textbf{Блинная} & \textbf{Гномья} & \textbf{Быстрая} & \textbf{Ожидаемый} \\
            \hline
            $[1, 2, 3, 4, 5]$ & $[1, 2, 3, 4, 5]$ & $[1, 2, 3, 4, 5]$ & $[1, 2, 3, 4, 5]$ & $[1, 2, 3, 4, 5]$ \\
            \hline
            $[5, 4, 3, 2, 1]$ & $[1, 2, 3, 4, 5]$ & $[1, 2, 3, 4, 5]$ & $[1, 2, 3, 4, 5]$ & $[1, 2, 3, 4, 5]$ \\
            \hline
            $[3, 8, 6, 4, 1]$ & $[1, 3, 4, 6, 8]$ & $[1, 3, 4, 6, 8]$ & $[1, 3, 4, 6, 8]$ & $[1, 3, 4, 6, 8]$ \\
            \hline
            $[2, 2, 2, 2]$ & $[2, 2, 2, 2]$ & $[2, 2, 2, 2]$ & $[2, 2, 2, 2]$ & $[2, 2, 2, 2]$ \\
            \hline
            $[2, 1]$ & $[1, 2]$ & $[1, 2]$ & $[1, 2]$ & $[1, 2]$ \\
            \hline
            $[7]$ & $[7]$ & $[7]$ & $[7]$ & $[7]$ \\
            \hline
        \end{tabular}
	\end{center}
\end{table}

\section*{Вывод}
\addcontentsline{toc}{section}{Вывод}

Были реализованы алгоритмы сортировки (блинная, быстрая, гномья).
Проведено тестирование реализованных алгортимов.
