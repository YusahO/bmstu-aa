\chapter{Аналитическая часть}

В данном разделе будут рассмотрены алгоритмы блинной, быстрой и гномьей сортировки.

\section{Блинная сортировка}

Единственная операция, допустимая в алгоритме~--- переворот элементов последовательности до какого-либо индекса.
В отличие от традиционных алгоритмов, в которых минимизируют количество сравнений, в блинной сортировке требуется сделать как можно меньше переворотов~\cite{pancake-sort}.

\section{Быстрая сортировка}

Процесс сортировки массива $A[p..r]$ алгоритмом быстрой сортировки состоит из трех этапов:
\begin{enumerate}
    \item массив $A[p..r]$ разбивается на два (возможно, пустых) подмассива $A[p..q-1]$ и $A[q+1..r]$, таких, что каждый элемент $A[p..q-1]$ меньше или равен $A[q]$, который, в свою очередь, не превышает любой элемент подмассива $A[q+1..r]$; индекс $q$ вычисляется в ходе процедуры разбиения;
    \item подмассивы $A[p..q-1]$ и $A[q+1..r]$ сортируются с помощью рекурсивного вызова процедуры быстрой сортировки;
    \item поскольку подмассивы сортируются на месте, весь массив $A[p..r]$ уже оказывается отсортированным~--- никаких дополнительных операций объединений подмассивов не требуется~\cite{qsort-book}.
\end{enumerate}

Элемент $A[q]$ также называется опорным элементом. 
В разных реализациях он может выбираться по-разному, например опорным элементом может быть выбран первый элемент массива, средний, случайный или медиана первого, среднего и последних элементов.

\section{Гномья сортировка}

Алгоритм ищет первую пару соседних элементов, которые находятся в неправильном порядке, и затем меняет их местами. 
Он пользуется тем фактом, что после обмена элементов может появиться новая пара, нарушающая порядок, только до или после переставленных элементов.
Алгоритм не проверяет, отсортированы ли элементы после текущей позиции, поэтому необходимо лишь проверить порядок до переставленных элементов~\cite{gnome-sort}.

\section*{Вывод}

В данном разделе были рассмотрены алгоритмы блинной, быстрой и гномьей сортировок.