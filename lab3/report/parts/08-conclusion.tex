\chapter*{Заключение}
\addcontentsline{toc}{chapter}{Заключение}

В результате выполнения лаботарторной работы по исследованию алгоритмов сортировок решены следующие задачи:
\begin{enumerate}
    \item описаны алгоритмы сортировки;
    \item разработаны и реализованы соответствующие алгоритмы;
    \item создан программный продукт, позволяющий протестировать реализованные алгоритмы;
    \item проведена оценка трудоемкостей алгоритмов сортировки:
    \begin{itemize}
        \item трудоемкость блинной сортировки в худшем случае~--- $O(N^2)$, в лучшем~--- $O(N^2)$;
        \item трудоемкость быстрой сортировки в худшем случае~--- $O(N^2)$, в лучшем~--- $O(N\log_2{N})$;
        \item трудоемкость гномьей сортировки в худшем случае~--- $O(N^2)$, в лучшем~--- $O(N)$;
    \end{itemize}
    \item проведен сравнительный анализ процессорного времени выполнения реализованных алгоритмов:
    \begin{itemize}
        \item алгоритм быстрой сортировки показал наибольшую скорость выполнения на упорядоченных по убыванию и неупорядоченных массивах;
        \item на массивах, упорядоченных по возрастанию, наиболее эффективным оказался алгоритм гномьей сортировки, однако в остальных случаях этот алгоритм показал наименьшую производительность;
        \item на неупорядоченных массивах скорость работы быстрой сортировки выше, чем на упорядоченных.
    \end{itemize}
    \item выполнена теоретическая оценка объема затрачиваемой памяти каждым из реализованных алгоритмов: алгоритм гномьей сортировки является наименее ресурсозатратным;
    алгоритм быстрой сортировки, напротив, требует больше всего памяти, что объясняется тем, что алгоритм рекурсивный, и на каждый вызов функции требуется выделение памяти на стеке для сохранения информации, связанной с этим вызовом.
\end{enumerate}