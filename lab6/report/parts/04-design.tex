\chapter{Конструкторская часть}

В данном разделе будут описаны требования к программному обеспечению и представлены схемы алгоритма полного перебора и муравьиного алгоритма.

\section{Требования к программному обеспечению}

К программе предъявлен ряд требований:
\begin{enumerate}
	\item наличие интерфейса для выбора действий;
	\item возможность сохранения сгенерированной матрицы смежности в файл и загрузки готовых матриц из файла;
	\item возможность выбора алгоритма решения задачи коммивояжера. 
\end{enumerate}

\section{Описание используемых типов данных}

При реализации алгоритмов будут использованы следующие типы данных:
\begin{itemize}
	\item размер матрицы смежности~--- целое число;
	\item имя файла~--- строка;
	\item коэффициенты $\alpha$, $\beta$, $k\_evaporation$, $elite\_deposit$~--- вещественные числа;
	\item матрица смежности~--- матрица вещественных чисел.
\end{itemize}

\section{Разработка алгоритмов}

На рисунке~\ref{img:scheme_brute_force} представлена схема алгоритма полного перебора, а на рисунке~\ref{img:scheme_ants}~--- схема муравьиного алгоритма поиска путей.

\includeimage
	{scheme_brute_force}
	{f}
	{H}
	{0.75\textwidth}
	{Схема алгоритма полного перебора}
	
\includeimage
	{scheme_ants}
	{f}
	{H}
	{0.8\textwidth}
	{Схема муравьиного алгоритма поиска путей}
	
На рисунке~\ref{img:scheme_cnp} представлена схема алгоритма выбора следующего города, на рисунке~\ref{img:scheme_find_pos}~--- схема алгоритма нахождения массива вероятностей переходов и на рисунке~\ref{img:scheme_up} показана схема алгоритма обновления матрицы феромонов.

\includeimage
	{scheme_cnp}
	{f}
	{H}
	{0.75\textwidth}
	{Схема алгоритма выбора следующего города}

\includeimage
	{scheme_find_pos}
	{f}
	{H}
	{0.8\textwidth}
	{Схема алгоритма нахождения массива вероятностей переходов}

\includeimage
	{scheme_up}
	{f}
	{H}
	{0.6\textwidth}
	{Схема алгоритма обновления матрицы феромонов}

\section*{Вывод}

В данном разделе были перечислены требования к программному обеспечению и построены схемы алгоритмов решения задачи коммивояжера.