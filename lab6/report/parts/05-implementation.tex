\chapter{Технологическая часть}

\section{Средства реализации}

В данной работе для реализации был выбран язык \textit{Python}~\cite{python-lang}, поскольку стандартная библиотека данного языка содержит все инструменты, требующиеся для реализации программного обеспечения.

\section{Средства замера времени}

Для выполнения измерений процессорного времени работы выполняемой программы использовалась функция \textit{process\_time} из модуля \textit{time}~\cite{python-lang-time}.
На листинге~\ref{lst:showcase.py} приведен пример замера процессорного времени с помощью функции \textit{process\_time}.

\includelistingpretty
	{showcase.py}
	{python}
	{Пример замера затраченного времени}

\section{Сведения о модулях программы}

Программа состоит из следующих модулей:
\begin{itemize}
	\item \texttt{main.py}~--- файл, содержащий точку входа в программу;
	\item \texttt{utils.py}~--- файл, содержащий вспомогательные функции;
	\item \texttt{bruteforce.py}~--- файл, содержащий реализацию алгоритма полного перебора;
	\item \texttt{ants.py}~--- файл, содержащий реализацию муравьиного алгоритма;
	\item \texttt{test.py}~--- файл, содержащий функции замера процессорного времени работы алгоритмов и подбора параметров.
\end{itemize}

\section{Реализация алгоритмов}

\section{Функциональные тесты}

В таблице~\ref{tbl:func-tests} приведены результаты функционального тестирования реализованных алгоритмов.
Все тесты пройдены успешно.

\begin{table}[H]
	\caption{Функциональные тесты}
	\label{tbl:func-tests}
	\centering
	\begin{tabular}{|c|c|c|}
		\hline
		Матрица смежности & Полный перебор & Муравьиный алгоритм \\ \hline
		$ \begin{pmatrix}
			0 &  4 &  2 &  1 & 7 \\
			4 &  0 &  3 &  7 & 2 \\
			2 &  3 &  0 & 10 & 3 \\
			1 &  7 & 10 &  0 & 9 \\
			7 &  2 &  3 &  9 & 0
		\end{pmatrix}$ &
		15, [0, 2, 4, 1, 3] &
		15, [0, 2, 4, 1, 3] \\ \hline
		$ \begin{pmatrix}
			0 & 1 & 2 \\
			1 & 0 & 1 \\
			2 & 1 & 0	
		\end{pmatrix}$ &
		4, [0, 1, 2] &
		4, [0, 1, 2] \\ \hline
		$ \begin{pmatrix}
			0 & 15 & 19 & 20 \\
			15 &  0 & 12 & 13 \\
			19 & 12 &  0 & 17 \\
			20 & 13 & 17 &  0
		\end{pmatrix}$ &
		64, [0, 1, 2, 3] &
		64, [0, 1, 2, 3] \\ \hline
	\end{tabular}
\end{table}

\section*{Вывод}

Были представлены листинги всех реализаций алгоритмов~--- полного перебора и муравьиного.
Также в данном разделе была приведена информации о выбранных средствах для разработки алгоритмов и сведения о модулях программы, проведено функциональное тестирование.