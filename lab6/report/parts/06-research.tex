\chapter{Исследовательская часть}

\section{Технические характеристики}

Технические характеристики устройства, на котором выполнялись замеры по времени:

\begin{itemize}
	\item процессор: AMD Ryzen 7 5800X (16) @ 3.800 ГГц.
	\item оперативная память: 32 ГБайт.
	\item операционная система: Manjaro Linux x86\_64 (версия ядра Linux~6.5.12-1-MANJARO).
\end{itemize}

Измерения проводились на стационарном компьютере.
Во время проведения измерений устройство было нагружено только системными приложениями.

\section{Демонстрация работы программы}

\section{Временные характеристики}

\section{Постановка исследования}

Автоматическая параметризация была проведена на двух классах данных --- \ref{par:class1} и \ref{par:class2}.
Алгоритм будет запущен для набора значений $\alpha, \rho \in (0, 1)$, $elite\_amt \in \{1, 5, 10\}$ и $elite\_depos \in \{0.2, 0.4, 0.8, 1\}$.

Итоговая таблица значений параметризации будет состоять из следующих колонок:
\begin{itemize}
	\item $\alpha$~--- коэффициент жадности;
	\item $\rho$~--- коэффициент испарения;
	\item $elite\_amt$~--- количество элитных муравьев;
	\item $elite\_depos$~--- коэффициент усиления феромонов у элитных муравьев;
	\item \textit{days}~--- количество дней жизни колонии муравьёв;
	\item \textit{Result}~--- эталонный результат, полученный методом полного перебора для проведения данного исследования;
	\item \textit{Mistake}~--- разность полученного основанным на муравьином алгоритме методом значения и эталонного значения на данных значениях параметров, показатель качества решения.
\end{itemize}

Цель исследования~--- определить комбинацию параметров, которые позволяют решить задачу наилучшим образом для выбранного класса данных.
Качество решения зависит от количества дней и погрешности измерений.

\subsection{Класс данных 1}
\label{par:class1}

Класс данных 1 представляет собой матрицу смежности размером 10 элементов (небольшой разброс значений: от 1 до 2), которая представлена в~(\ref{eqn:kd1}).

\begin{equation}
	\label{eqn:kd1}
	K_{1} = \begin{pmatrix}
		0 & 1 & 1 & 2 & 2 & 1 & 1 & 1 & 2 \\ 
		1 & 0 & 1 & 2 & 1 & 1 & 2 & 1 & 1 \\ 
		1 & 1 & 0 & 2 & 2 & 1 & 1 & 2 & 2 \\ 
		2 & 2 & 2 & 0 & 1 & 2 & 1 & 2 & 2 \\ 
		2 & 1 & 2 & 1 & 0 & 2 & 2 & 1 & 1 \\ 
		1 & 1 & 1 & 2 & 2 & 0 & 1 & 1 & 2 \\ 
		1 & 2 & 1 & 1 & 2 & 1 & 0 & 2 & 2 \\ 
		1 & 1 & 2 & 2 & 1 & 1 & 2 & 0 & 2 \\ 
		2 & 1 & 2 & 2 & 1 & 2 & 2 & 2 & 0 
	\end{pmatrix}
\end{equation}

\subsection{Класс данных 2}
\label{par:class2}

\begin{equation}
	\label{eqт:kd2}
	K_{2} = \begin{pmatrix}
		0 & 9271 & 8511 & 2010 & 1983 & 7296 & 7289 & 3024 & 1011 \\
		9271 & 0 & 7731 & 4865 & 5494 & 6812 & 4755 & 7780 & 7641 \\
		8511 & 7731 & 0 & 1515 & 9297 & 7506 & 5781 & 5804 & 7334 \\
		2010 & 4865 & 1515 & 0 & 3662 & 9597 & 2876 & 8188 & 9227 \\
		1983 & 5494 & 9297 & 3662 & 0 & 8700 & 4754 & 7445 & 3834 \\
		7296 & 6812 & 7506 & 9597 & 8700 & 0 & 4216 & 5553 & 8215 \\
		7289 & 4755 & 5781 & 2876 & 4754 & 4216 & 0 & 4001 & 4715 \\
		3024 & 7780 & 5804 & 8188 & 7445 & 5553 & 4001 & 0 & 9522 \\
		1011 & 7641 & 7334 & 9227 & 3834 & 8215 & 4715 & 9522 & 0 
	\end{pmatrix}
\end{equation}

\subsection{Класс данных 3}

\section{Вывод}