\chapter*{ВВЕДЕНИЕ}
\addcontentsline{toc}{chapter}{ВВЕДЕНИЕ}

Целью данной лабораторной работы является параметризация метода решения задачи коммивояжера на основе муравьиного метода.

Для достижения поставленной цели, необходимо решить следующие задачи:
\begin{enumerate}
    \item описать задачу коммивояжера;
    \item описать методы решения задачи коммивояжера: метод полного перебора и метод на основе муравьиного алгоритма;
    \item реализовать данные алгоритмы;
    \item сравнить по времени метод полного перебора и метод на основе муравьиного алгоритма
\end{enumerate}

Выданный индивидуальный вариант для выполнения лабораторной работы:
\begin{itemize}
	\item ориентированный граф;
	\item с элитными муравьями;
	\item незамкнутый маршрут;
	\item города России XVI века;
	\item зимой можно ходить по рекам в обе стороны за равную цену, летом по течению в 2 раза быстрее, против~--- в 4 раза медленнее.
\end{itemize}