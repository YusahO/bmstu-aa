\chapter*{ЗАКЛЮЧЕНИЕ}
\addcontentsline{toc}{chapter}{ЗАКЛЮЧЕНИЕ}

В ходе лабораторной работы поставленная ранее цель была достигнута, решены все задачи, а именно:
\begin{enumerate}
	\item описана задача коммивояжера;
	\item описаны методы решения задачи коммивояжера: метод полного перебора и метод на основе муравьиного алгоритма;
	\item реализованы данные алгоритмы;
	\item проведено сравнение по времени реализованных алгоритмов и сделаны следующие выводы:
	\begin{itemize}
		\item 
	\end{itemize}
	\item проведена параметризация муравьиного алгоритма и сделаны следующие выводы:
	\begin{itemize}
		\item для класса данных 3 было получено, что наилучшим образом алгоритм работает на значениях параметров, которые представлены далее:
		\begin{itemize}[label=---]
			\item $\alpha = 0.1, \rho = 0.3, 0.7$;
			\item $\alpha = 0.2, \rho = 0.1, 0.2, 0.5, 0.8$;
			\item $\alpha = 0.3, \rho = 0.1, 0.2$;
			\item $\alpha = 0.4, \rho = 0.5$;
			\item $\alpha = 0.5, \rho = 0.2$;
			\item $\alpha = 0.6, \rho = 0.2, 0.3, 0.4$.
		\end{itemize} 
		
		Для этого класса данных рекомендуется использовать данные параметры.
	\end{itemize}
\end{enumerate}