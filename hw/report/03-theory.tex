\chapter{Введение}

\section{Задание}

Описать четырьмя графовыми моделями (ГУ, ИГ, ОИ, ИИ) последовательный алгоритм либо фрагмент алгоритма, содержащий от 15 значащих строк кода и от двух циклов, один из которых является вложенным в другой.

\textbf{Вариант 21}: реализуемый алгоритм~--- кластеризация нечетким алгоритмом c-средних.

\section{Графовые модели программы}

Программа представлена в виде графа: набор вершин и множество соединяющих их направленных дуг.

\begin{enumerate}
    \item \textbf{Вершины}: процедуры, циклы, линейные участки, операторы, итерации циклов, срабатывание операторов и т. д.
    \item \textbf{Дуги} отражают связь (отношение между вершинами). 
\end{enumerate}

Выделяют 2 типа отношений:
\begin{enumerate}
    \item операционное отношение~--- по передаче управления;
    \item информационное отношение~--- по передаче данных.
\end{enumerate}

Граф управления:
\begin{enumerate}
    \item \textbf{Вершины}~--- операторы.
    \item \textbf{Дуги}~--- операционные отношения.
\end{enumerate}

Информационный граф:
\begin{enumerate}
    \item \textbf{Вершины}~--- операторы.
    \item \textbf{Дуги}~--- информационные отношения.
\end{enumerate}

Операционная история:
\begin{enumerate}
    \item \textbf{Вершины}~--- срабатывание операторов.
    \item \textbf{Дуги}~--- операционные отношения.
\end{enumerate}

Информационная история:
\begin{enumerate}
    \item \textbf{Вершины}~--- срабатывание операторов.
    \item \textbf{Дуги}~--- информационные отношения.
\end{enumerate}

