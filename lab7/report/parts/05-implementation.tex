\chapter{Технологическая часть}

В данном разделе описаны средства реализации программного обеспечения, а также листинги и функциональные тесты.

\section{Средства реализации}

В качестве языка программирования, используемого при написании данной лабораторной работы, был выбран C++~\cite{cpp-lang}, так как в нем имеется контейнер \texttt{std::vector}, представляющий собой динамический массив данных произвольного типа, и библиотека \texttt{<ctime>}~\cite{cpp-ctime}, позволяющая производить замеры процессорного времени.
Также выбранный язяк программирования предоставляет возможность работы с:
\begin{itemize}
	\item потоками (класс \texttt{thread}~\cite{cpp-thread});
	\item мьютексами (класс \texttt{mutex}~\cite{cpp-mutex});
\end{itemize}

Обработка файлов MIDI производилась с помощью библиотеки MidiFile~\cite{midifile}.

\section{Сведения о модулях программы}

Данная программа разбита на следующие модули:

\begin{itemize}
	\item \texttt{main.cpp}~--- файл, содержащий точку входа в программу;
	\item \texttt{algorithms.cpp}~--- файл, содержащий реализации алгоритмов, используемых на различных стадиях конвейера;
	\item \texttt{ts\_queue.cpp}~--- файл, содержащий реализацию потокобезопасной очереди;
	\item \texttt{pipeline.cpp}~--- файл, содержащий функции конвейерной обработки;
	\item \texttt{utils.cpp}~--- файл, содержащий вспомогательные функции;
	\item \texttt{measure.cpp}~--- файл, содержащий функции, замеряющие процессорное время выполнения реализуемых алгоритмов.
\end{itemize}

\section{Реализация алгоритмов}

На листинге~\ref{lst:algos.cpp} представлены реализации алгоритмов извлечения нот из файлов MIDI, извлечения биграмм и кластеризации c-средних.

\includelistingpretty
	{algos.cpp}
	{c++}
	{Реализации алгоритмов извлечения нот из файлов MIDI, извлечения биграмм и кластеризации c-средних}

На листинге~\ref{lst:linear_pipe.cpp} представлена реализация линейного алгоритма обработки набора файлов MIDI.

\includelistingpretty
	{linear_pipe.cpp}
	{c++}
	{Реализация линейного алгоритма обработки набора файлов MIDI}

На листинге~\ref{lst:conc_pipe.cpp} представлена реализация конвейерного алгоритма обработки набора файлов MIDI.

\includelistingpretty
	{conc_pipe.cpp}
	{c++}
	{Реализация конвейерного алгоритма обработки набора файлов MIDI}
	
\section*{Вывод}

В данном разделе были рассмотрены средства реализации, а также представлен листинг реализаций линейного и конвейерного алгоритмов обработки набора файлов MIDI.