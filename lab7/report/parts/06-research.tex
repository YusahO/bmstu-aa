\chapter{Исследовательская часть}

В данном разделе приведены технические характеристики устройства, на котором проводилось измерение времени работы программного обеспечения, а также результаты замеров времени.

\section{Технические характеристики}

Технические характеристики устройства, на котором выполнялись замеры по времени:

\begin{itemize}
	\item процессор: AMD Ryzen 7 5800X @ 3.800 ГГц, 8 физ. ядер, 16 лог. ядер;
	\item оперативная память: 32 ГБайт.
	\item операционная система: Manjaro Linux x86\_64 (версия ядра Linux~6.5.12-1-MANJARO).
\end{itemize}

Измерения проводились на стационарном компьютере.
Во время проведения измерений устройство было нагружено только системными приложениями.

\section{Демонстрация работы программы}

На рисунке \ref{img:prog_demo} продемонстрирована работа программы для случая, когда пользователь выбрал пункт 2 <<Запустить последовательную обработку датасета файлов MIDI>>, выбрал файл с описанием датасета под номером 1 и ввел следующие значения:
\begin{itemize}
	\item количество заявок~--- $10$,
	\item число кластеров~--- $3$,
	\item значение показателя нечеткости~--- $2$,
	\item значение порога сходимости~--- $1$,
	\item максимальное к-во итераций~--- $10$.
\end{itemize}

\includeimage
	{prog_demo}
	{f}
	{H}
	{1\textwidth}
	{Демонстрация работы программы}

\section{Временные характеристики}

Исследование временных характеристик реализуемых алгоритмов производилось 2 раза:
\begin{enumerate}
	\item при изменении числа заявок от 5 до 50 с шагом 5 для датасета, состоящего из двух файлов MIDI;
	\item при изменения размера датасета от 1 до 10 файлов MIDI при 10 заявках.
\end{enumerate}

Для кластеризации были выбраны следующие параметры:
\begin{itemize}
	\item число кластеров~--- $5$,
	\item значение показателя нечеткости~--- $2$,
	\item значение порога сходимости~--- $1$,
	\item максимальное к-во итераций~--- $1000$.
\end{itemize}

\begin{filecontents*}{parts/time_queries.csv}
	5,571257.46,2626.42
	10,887551.33,4886.41
	15,1680618.36,7333.44
	20,2133323.53,9746.04
	25,2619305.55,12219.58
	30,3117679.17,14619.90
	35,3427449.75,17047.36
	40,4206998.96,19596.30
	45,4854466.86,21948.68
	50,5699781.42,24419.38
\end{filecontents*}

\begin{filecontents*}{parts/time_files.csv}
	1,280633.58,2755.60
	2,559750.83,5334.26
	3,839651.58,8110.10
	4,1117778.79,10803.40
	5,1395546.68,13297.58
	6,1674180.20,16128.27
	7,1953854.11,18811.78
	8,2234085.88,21565.09
	9,2511400.88,24301.44
	10,2791956.47,27051.17
\end{filecontents*}

На рисунке~\ref{img:time_queries} приведены результаты измерения времени работы алгоритмов линейной и конвейерной обработки датасетов файлов MIDI при варьировании числа заявок; таблица~\ref{tbl:time_queries} содержит данные, по которой был построен данный график.

\includeimage
	{time_queries}
	{f}
	{H}
	{1\textwidth}
	{Сравнения времени работы алгоритмов линейной и конвейерной обработки датасетов файлов MIDI при варьировании числа заявок}
	
\begin{table}[H]
	\centering
	\caption{Результаты измерения времени работы реализуемых алгоритмов при варьировании числа заявок}
	\label{tbl:time_queries}
	\begin{tabular}{|c|r|r|}
		\hline
		~ & \multicolumn{2}{c|}{Время (мс)} \\
		\cline{2-3}
		К-во файлов (шт.) & Линейный & Конвейерный \\ \hline
		\csvreader[
		separator=comma,
		head=false,
		late after line = \\\hline
		]{parts/time_files.csv}{}{% 
			\csvcoli & \csvcolii & \csvcoliii  
		}
	\end{tabular}
\end{table}

На рисунке~\ref{img:time_files} приведены результаты измерения времени работы алгоритмов линейной и конвейерной обработки датасетов файлов MIDI при варьировании числа файлов в датасете; таблица~\ref{tbl:time_files} содержит данные, по которой был построен данный график.

\includeimage
	{time_files}
	{f}
	{H}
	{1\textwidth}
	{Сравнения времени работы алгоритмов линейной и конвейерной обработки датасетов файлов MIDI при варьировании числа файлов в датасете}
	
\begin{table}[H]
	\centering
	\caption{Результаты измерения времени работы реализуемых алгоритмов при варьировании числа заявок}
	\label{tbl:time_files}
	\begin{tabular}{|c|r|r|}
		\hline
		~ & \multicolumn{2}{c|}{Время (мс)} \\
		\cline{2-3}
		К-во заявок (шт.) & Линейный & Конвейерный \\ \hline
		\csvreader[
		separator=comma,
		head=false,
		late after line = \\\hline
		]{parts/time_queries.csv}{}{% 
			\csvcoli & \csvcolii & \csvcoliii  
		}
	\end{tabular}
\end{table}

\section{Вывод}

В результате исследования реализуемых алгоритмов по времени выполнения можно сделать следующие выводы:
\begin{enumerate}
	\item при варьировании числа заявок линейный алгоритм обработки датасета, состоящем из двух файлов MIDI, выполнялся дольше конвейерного в среднем в $214.5$ раз;
	\item при варьировании числа файлов линейный алгоритм обработки датасета при 10 заявках выполнялся дольше конвейерного в среднем в $103.7$ раз.
\end{enumerate}
