\chapter{Аналитическая часть}

В данном разделе приведена информация, касающаяся алгоритма бинарного поиска.

\section{Алгоритм бинарного поиска}

Алгоритм бинарного поиска является алгоритмом поиска значения в отсортированном массиве данных.

Алгоритм бинарного поиска начинает сравнение целевого значения с элементом в середине массива.
Если они не равны, половина массива, в которой целевое значение не может находиться, удаляется, и поиск продолжается в оставшейся половине.
Затем цикл повторяется: снова берется элемент в середине оставшейся половины, сравнивается с целевым значением, и процесс продолжается до тех пор, пока не будет найдено целевое значение.
Если поиск завершается с пустой оставшейся половиной, значит, целевое значение отсутствует в массиве.

В лучшем случае искомое значение находится в середине массива~--- тогда алгоритм отрабатывает за 1 итерацию.

В худшем случае искомый элемент отсутствует в массиве~--- тогда количество итераций равно $ceil(\log_2(n) + 1)$.