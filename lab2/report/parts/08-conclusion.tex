\chapter*{Заключение}
\addcontentsline{toc}{chapter}{Заключение}

В результате выполнения лаботарторной работы по исследованию алгоритмов умножения матриц решены следующие задачи:
\begin{enumerate}
    \item описаны алгоритмы умножения матриц;
    \item разработаны и реализованы соответствующие алгоритмы;
    \item создан программный продукт, позволяющий протестировать реализованные алгоритмы;
    \item проведен сравнительный анализ процессорного времени выполнения реализованных алгоритмов:
    \begin{itemize}
        \item оптимизированный алгоритм Винограда оказался самым эффективным по времени независимо от размерности входных матриц;
        \item время работы классического алгоритма умножения и оптимизированного и неоптимизированного алгоритмов Винограда на матрицах нечетного размера больше, чем на матрицах четного размера;
        для алгоритма Винограда меньшая скорость работы на матрицах нечетного размера объясняется необходимостью дополнительных вычислений крайних строк и столбцов в результирующей матрице;
        \item алгоритм Штрассена показал наименьшую производительность среди всех алгоритмов, исследуемых на матрицах, размер которых равен степени двойки;
        низкая производительность алгоритма обуславливается необходимостью выполнения дополнительных операций сложения/вычитания, разбиения/слияния матриц;
    \end{itemize}
    \item выполнена теоретическая оценка объема затрачиваемой памяти каждым из реализованных алгоритмов: стандартный алгоритм умножения матриц является наименее ресурсозатратным из всех реализованных алгоритмов;
    самым же требовательным по памяти оказался алгоритм Штрассена за счет использования вспомогательных подматриц для выполнения рассчетов и рекурсивных вызовов;
    оптимизированный алгоритм Винограда использует больше ресурсов по сравнению с неоптимизированным, за счет предвычисления некоторых выражений.
\end{enumerate}