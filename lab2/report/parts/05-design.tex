\chapter{Конструкторская часть}

В данном разделе будут приведены схемы алгоритмов умножения матриц, описание используемых типов данных и структуры программного обеспечения.

\section{Требования к ПО}

К программе предъявлен ряд требований:

\begin{itemize}
    \item на вход программе подаются две матрицы, каждая записана в отдельном текстовом файле;
    \item результатом умножения является матрица, выводимая на экран;
    \item программа должна позволять производить измерения процессорного времени, затрачиваемого на выполнение реализуемых алгоритмов.
\end{itemize}

\section{Разработка алгоритмов}

На рисунке \ref{fig:scheme-classic} представлена схема классического алгоритма умножения матриц.

На рисунках \ref{fig:scheme-winograd-pt1} -- \ref{fig:scheme-winograd-pt2} приведена схема умножения матриц алгоритмом Винограда.

На рисунках \ref{fig:scheme-winograd-opt-pt1} -- \ref{fig:scheme-winograd-opt-pt2} показана схема умножения матриц оптимизированным алгоритмом Винограда.

На рисунке \ref{fig:scheme-strassen} представлена схема алгоритма Штрассена.

\begin{figure}[H]
    \centering
    \includesvg[width=0.8\textwidth]{images/scheme_classic.svg}
    \caption{Схема классического алгоритма умножения матриц}
    \label{fig:scheme-classic}
\end{figure}

\begin{figure}[H]
    \centering
    \includesvg[width=0.5\textwidth]{images/scheme_winograd_pt1.svg}
    \caption{Схема алгоритма Винограда}
    \label{fig:scheme-winograd-pt1}
\end{figure}

\begin{figure}[H]
    \centering
    \includesvg[width=0.6\textwidth]{images/scheme_winograd_pt2.svg}
    \caption{Схема алгоритма Винограда (продолжение)}
    \label{fig:scheme-winograd-pt2}
\end{figure}

\begin{figure}[H]
    \centering
    \includesvg[width=0.5\textwidth]{images/scheme_winograd_opt_pt1.svg}
    \caption{Схема оптимизированного алгоритма Винограда}
    \label{fig:scheme-winograd-opt-pt1}
\end{figure}

\begin{figure}[H]
    \centering
    \includesvg[width=0.6\textwidth]{images/scheme_winograd_opt_pt2.svg}
    \caption{Схема оптимизированного алгоритма Винограда (продолжение)}
    \label{fig:scheme-winograd-opt-pt2}
\end{figure}

\begin{figure}[H]
    \centering
    \includesvg[width=0.5\textwidth]{images/scheme_strassen.svg}
    \caption{Схема алгоритма Штрассена}
    \label{fig:scheme-strassen}
\end{figure}

\section{Описание используемых типов данных}

При реализации алгоритмов будут использованы следующие типы данных:

\begin{itemize}
    \item \textit{матрица}~--- двумерный массив значений типа \texttt{int}.
\end{itemize}

\section{Модель вычисления для проведения оценки трудоемкости}

Введем модель вычислений, которая потребуется для определения трудоемкости каждого отдельного взятого алгоритма умножения матриц.

\begin{enumerate}
    \item Трудоемкость базовых операций имеет:
    \begin{itemize}
        \item значение 1 для операций:
        \begin{equation}
            \begin{gathered}
                +, -, =, +=, -=, ==, !=, <, >, <=, >=, \\ 
                \text{[]}, ++, --, \&\&, ||, >>, <<, \&, |
            \end{gathered}
        \end{equation}
        \item значение 2 для операций:
        \begin{equation}
            *, /, \%, *=, /=, \%=.
        \end{equation}
    \end{itemize}
    \item Трудоемкость условного оператора:
    \begin{equation}
        f_{\text{if}} =
        \begin{cases}
            \min({f_1, f_2}), & \text{лучший случай} \\
            \max({f_1, f_2}), & \text{худший случай}.
        \end{cases}
    \end{equation}
    \item Трудоемкость цикла
    \begin{equation}
        \begin{gathered}
            f_{\text{for}} = f_{\text{инициализация}} + f_{\text{сравнение}} + \\
            + M_{\text{итераций}} \cdot (f_{\text{тело}} + f_{\text{инкремент}} + f_{\text{сравнение}}).
        \end{gathered} 
    \end{equation}
    \item Трудоемкость передачи параметра в функцию и возврат из нее равен 0.
\end{enumerate}

\section{Трудоемкость алгоритмов}

Далее будут приведены рассчеты трудоемкости реализуемых алгоритмов умножения матриц.

Пусть имеется две матрицы:

\begin{enumerate}
    \item $A$ размером $M \times N$,
    \item $B$ размером $N \times P$.
\end{enumerate}

\subsection{Классический алгоритм}

Для стандартного алгоритма умножения матриц трудоемкость будет составлена из:

\begin{itemize}
    \item цикла по $i \in [1 \ldots M]$, трудоемкость которого $f = 2 + M \cdot (2 + f_\text{тело})$;
    \item цикла по $j \in [1 \ldots P]$, трудоемкость которого $f = 2 + P \cdot (2 + f_\text{тело})$;
    \item цикла по $k \in [1 \ldots N]$, трудоемкость которого $f = 2 + N \cdot (2 + 12)$;
\end{itemize}

Поскольку трудоемкость стандартного алгоритма равна трудоемкости внешнего цикла, то

\begin{equation}
    \begin{gathered}
        f_{Standard} = 2 + M \cdot (2 + 2 + P \cdot (2 + 2 + N \cdot (3 + 3 + (2 + 2 + 2)))) = \\
        = 2 + 4M + 4MN + 14MNP \approx 14MNP = O(N^3)
    \end{gathered}
\end{equation}

\subsection{Алгоритм Винограда}

Трудоемкость алгоритма Винограда складывается из:
\begin{itemize}
    \item трудоемкости создания и инициализации массивов $RowFactors$ и $ColFactors$:
    \begin{equation}
        \label{eqn:f-arrs}
        f_{arrs} = f_{RowFactors} + f_{ColFactors}
    \end{equation}
    \item трудоемкость заполнения массива $RowFactors$:
    \begin{equation}
        \begin{gathered}
            f_{RowFactors} = 2 + M \cdot (2 + 4 + \frac{N}{2} \cdot (4 + 6 + 1 + 2 + 3 \cdot 2)) = \\
            2 + 6M + \frac{19MN}{2};
        \end{gathered}
    \end{equation}
    \item трудоемкость заполнения массива $ColFactors$:
    \begin{equation}
        \begin{gathered}
            f_{ColFactors} = 2 + P \cdot (2 + 4 + \frac{N}{2} \cdot (4 + 6 + 1 + 2 + 3 \cdot 2)) = \\
            2 + 6P + \frac{19NP}{2};
        \end{gathered}
    \end{equation}
    \item трудоемкость цикла умножения для четных размеров:
    \begin{equation}
        \begin{gathered}
            f_{mul} = 2 + M \cdot (4 + P \cdot (13 + 32\frac{N}{2})) = 2 + 4M + \\
            + 13MP + \frac{32MNP}{2} = 2 + 4M + 13MP + 16MNP;
        \end{gathered}
    \end{equation}
    \item трудоемкость цикла, выполняемого в случае нечетных размеров матрицы:
    \begin{equation}
        f_{oddLoop} = 3 + 
        \begin{cases}
            0, & \text{четная} \\
            2 + M \cdot (4 + P \cdot (2 + 14)), & \text{иначе.}
        \end{cases}
    \end{equation}
\end{itemize}
Таким образом, для нечетного размера матрицы имеем:
\begin{multline}
    f_{odd} = f_{arrs} + f_{RowFactors} + f_{ColFactors} + f_{mul} + f_{oddLoop} \approx \\ \approx 16MNP = O(N^3);
\end{multline}
для четного:
\begin{multline}
    f_{even} = f_{arrs} + f_{RowFactors} + f_{ColFactors} + f_{mul} + f_{oddLoop} \approx \\ \approx 16MNP = O(N^3).
\end{multline}

\subsection{Оптимизированный алгоритм Винограда}

Оптимизация алгоритма Винограда осуществляется следующим образом:
\begin{itemize}
    \item операция $x = x + k$ заменяется на операцию $x += k$;
    \item операция $x \cdot 2$ заменяется на $x << 1$;
    \item некоторые значения для алгоритма вычисляются заранее.
\end{itemize}

Тогда трудоемкость алгоритма Винограда c примененными оптимизациями складывается из:
\begin{itemize}
    \item трудоемкости предвычисления значения $\frac{N}{2}$, равной 3;
    \item трудоемкости $f_{arrs}$ (\ref{eqn:f-arrs}) создания и инициализации массивов $RowFactors$ и $ColFactors$;
    \item трудоемкость заполнения массива $RowFactors$:
    \begin{equation}
        \begin{gathered}
            f_{RowFactors} = 2 + M \cdot (2 + 2 + \frac{N}{2} \cdot (2 + 2 + 2 + 7)) = \\
            2 + 2M + \frac{13MN}{2};
        \end{gathered}
    \end{equation}
    \item трудоемкость заполнения массива $ColFactors$:
    \begin{equation}
        \begin{gathered}
            f_{ColFactors} = 2 + P \cdot (2 + 2 + \frac{N}{2} \cdot (2 + 2 + 2 + 7)) = \\
            2 + 2P + \frac{13NP}{2};
        \end{gathered}
    \end{equation}
    \item трудоемкость цикла умножения для четных размеров:
    \begin{equation}
        \begin{gathered}
            f_{mul} = 2 + M \cdot (4 + P \cdot (4 + 2 + \frac{N}{2} \cdot (2 + 2 + 3 + 6 + 2 + 6))) = \\
            = 2 + 4M + 13MP + \frac{32MNP}{2} = 2 + 4M + 13MP + 16MNP;
        \end{gathered}
    \end{equation}
    \item трудоемкость цикла, выполняемого в случае нечетных размеров матрицы:
    \begin{equation}
        f_{oddLoop} = 3 + 
        \begin{cases}
            0, & \text{четная} \\
            2 + M \cdot (4 + P \cdot (2 + 11)), & \text{иначе.}
        \end{cases}
    \end{equation}
\end{itemize}
Таким образом, для нечетного размера матрицы имеем:
\begin{multline}
    f_{odd} = f_{arrs} + f_{RowFactors} + f_{ColFactors} + f_{mul} + f_{oddLoop} \approx \\ \approx \frac{19MNP}{2} = O(N^3);
\end{multline}
для четного:
\begin{multline}
    f_{even} = f_{arrs} + f_{RowFactors} + f_{ColFactors} + f_{mul} + f_{oddLoop} \approx \\ \approx \frac{19MNP}{2} = O(N^3).
\end{multline}

\subsection{Алгоритм Штрассена}

Пусть матрицы $A$ и $B$ имеют размер $n \times n$, где $n$~--- степень 2.

Если $M(n)$~---  количество умножений, выполняемых алгоритмом Штрассена для умножения двух матриц размером $n \times n$, то рекуррентное соотношение для $M(n)$ будет иметь следующий вид~\cite{strassen-elib}:

\begin{equation}
    M(n) = 7M \left (\frac{n}{2} \right ), n > 1, M(1) = 1.
\end{equation}
Поскольку $n = 2^k$:
\begin{equation}
    \begin{gathered}
        M(2^k) = 7M(2^{k - 1}) = 7 \cdot [7M(2^{k - 2})] = 7^2 \cdot M(2^{k-2}) = \ldots = \\
        = 7^i \cdot M(2^{k-i}) = \ldots = 7^k \cdot M(2^{k-k}) = 7^k.
    \end{gathered}
\end{equation}
Так как $k = \log_2{(n)}$:
\begin{equation}
    M(n) = 7^{\log_2{(7)}} = n^{\log_2{(7)}} \approx n^{2.807}.
\end{equation}

Рассмотрим количество сложений $A(n)$.
Для умножения двух матриц порядка $n > 1$ алгоритму требуется 7 умножений и 18 сложений матриц размером $\frac{n}{2} \times \frac{n}{2}$~\cite{strassen-elib}; при $n > 1$ сложений не требуется, так как задача вырождается в перемножение двух чисел, поэтому количество сложений оценивается следующим образом:

\begin{equation}
    A(n) = 7 \cdot A \left ( \frac{n}{2} \right ) + 18 \cdot \left ( \frac{n}{2} \right )^2, n > 1, A(1) = 0.
\end{equation}

Итоговая трудоемкость рассчитывается как:

\begin{equation}
    T(n) = A(n) + M(n).
\end{equation}

\section*{Вывод}
\addcontentsline{toc}{section}{Вывод}

В данном разделе на основе теоретических данных были построены схемы реализуемых алгоритмов умножения матриц.
Также была проведена оценка трудоемкостей алгоритмов.
В результате оценки алгоритма Винограда выяснилось, что оптимизированная версия в 1.6 раз менее трудоемка, чем неоптимизированная.