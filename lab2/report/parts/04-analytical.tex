\chapter{Аналитическая часть}

В данном разделе будут рассмотрены классический алгоритм умножения матриц, алгоритм Винограда (неоптимизированный и оптимизированный), а также алгоритм Штрассена.

\section{Матрица}
Матрицей типа (или размера) $m \times n$ называют прямоугольную числовую таблицу, состоящую из $m \cdot n$ чисел, которые расположены в $m$ строках и $n$ столбцах~\cite{about-matrix}.

Обозначаются:

\begin{equation}
	\begin{pmatrix}
		a_{11} & a_{12} & \ldots & a_{1n}\\
		a_{21} & a_{22} & \ldots & a_{2n}\\
		\vdots & \vdots & \ddots & \vdots\\
		a_{m1} & a_{m2} & \ldots & a_{mn}
	\end{pmatrix}
\end{equation}
или сокращенно $(a_{ij}), i = \overline{1,m}, j = \overline{1,n}$~\cite{about-matrix}.

Над матрицами возможны следующие операции:
\begin{itemize}
    \item сложение матриц одинакового размера;
    \item умножение матриц, количество столбцов первой матрицы равно количеству строк второй матрицы~\cite{about-matrix}.
\end{itemize}

\section{Классический алгоритм}

Пусть даны матрица $A = (a_{ij})$ размером $m \times n$ и матрица $B = (b_{ij})$ размером $n \times p$.
Произведением матриц $A$ и $B$ называют матрицу $C = (c_{ij})$ размером $m \times p$ с элементами
\begin{equation}
    \label{eqn:classic-mul}
    c_{ij} = \sum_{k=1}^{n}a_{ik}b_{kj},
\end{equation}
которую обозначают $C = AB$~\cite{about-matrix}.

Классический алгоритм реализует формулу (\ref{eqn:classic-mul}).

\section{Алгоритм Винограда}

Алгоритм Винограда является одним из самых эффективных алгоритмов умножения матриц, имея асимптотическую сложность $O(n^{2.3755})$~\cite{winograd-haskell}.

Рассмотрим два вектора $A = (a_1, \ldots, a_n)$ и $B = (b_1, \ldots, b_n)$.

Их скалярное произведение равно:
\begin{equation}
    \label{eqn:dot-classic}
    A \cdot B = \sum_{i=1}^{n}a_ib_i
\end{equation}
\begin{equation}
    \label{eqn:dot-winograd}
    A \cdot B = \sum_{i=1}^{\frac{n}{2}}(a_{2i} + b_{2i+1})(a_{2i+1} + b_{2i}) + \sum_{i=1}^{\frac{n}{2}}a_ia_{i+1} + \sum_{i=1}^{\frac{n}{2}}b_ib_{i+1}
\end{equation}

В выражении (\ref{eqn:dot-winograd}) выполняется большее количество вычислений, чем в выражении (\ref{eqn:dot-classic}), однако второе и третье слагаемые из (\ref{eqn:dot-winograd}) Виноград предложил вычислять предварительно~\cite{winograd-haskell}.
Таким образом удается уменьшить количество операций умножения, являющихся более трудоемкими, чем операции сложения.

Если обрабатываемые матрицы имеют нечетный размер, то неоюходимо дополнительно рассчитать произведения крайних строк и столбцов.

\section{Оптимизированный алгоритм Винограда}

Для программной реализации алгоритма Винограда существует несколько оптимизаций:
\begin{itemize}
    \item значение $d = \frac{n}{2}$, используемое в качестве ограничения цикла расчета второго и третьего слагаемых из соотношения (\ref{eqn:dot-winograd}) сохранить в переменную;
    \item заменить операцию умножения на 2 на операциб побитового сдвига влево на 1;
    \item при наличии операторов \texttt{+=, -=} в выбранном языке программирования использовать их при необходимости.
\end{itemize}

\section{Алгоритм Штрассена}

Пусть $A$ и $B$~--- матрицы размером $n \times n$, где $n$~--- степень числа 2.
Поделим эти матрицы на четыре части, пополам по вертикали и горизонтали, например
\begin{equation}
    \label{eqn:mat-split}
    A = 
    \begin{pmatrix}
        A_{11} & A_{12}\\
        A_{21} & A_{22}
    \end{pmatrix},
\end{equation}
где $A_{ij}$~--- подматрицы матрицы $A$, имеющие размер $\frac{n}{2} \times \frac{n}{2}$.

Пусть матрица $C$~--- результирующая матрица, элементы которой в случае выбранного разбиения матриц $A$ и $B$ будут равны
\begin{equation}
    \label{eqn:c-mat-from-split}
    \begin{aligned}
        C_{11} = A_{11}B_{11} + A_{12}B_{21},\\
        C_{12} = A_{11}B_{12} + A_{12}B_{22},\\
        C_{21} = A_{21}B_{11} + A_{22}B_{21},\\
        C_{22} = A_{21}B_{12} + A_{22}B_{22}.
    \end{aligned}
\end{equation}

Разбиение матриц $A_{ij}$, $B_{ij}$ выполняется рекурсивно до того момента, пока перемножение матриц не будет сведено к перемножению чисел~\cite{strassen-lect}.

При таком подходе для матрицы размером $2 \times 2$ количество операций умножения равно 8.
Количество умножений можно снизить до 7, используя алгоритм Штрассена~\cite{strassen-elib}.

Для выбранного разбиения необходимо ввести новые матрицы:

\begin{flalign}
    \label{eqn:aux-mat-strassen}
    \begin{aligned}
        M_1 &= (A_{12} - A_{22})(B_{21} + B_{22}),\\
        M_2 &= (A_{11} + A_{22})(B_{11} + B_{22}),\\
        M_3 &= (A_{11} - A_{21})(B_{11} + B_{12}),\\
        M_4 &= (A_{11} + A_{12})B_{22},\\
        M_5 &= A_{11}(B12 - B_{22}),\\
        M_6 &= A_{22}(B21 - B_{11}),\\
        M_7 &= (A_{21} + A_{22})B_{11}.
    \end{aligned}
\end{flalign}
Тогда $C_{ij}$ выражаются через эти матрицы~\cite{strassen-lect}:
\begin{flalign}
    \label{eqn:c-mat-strassen}
    \begin{aligned}
        C_{11} &= M_1 + M_2 - M_4 + M_6,\\
        C_{12} &= M_4 + M_5,\\
        C_{21} &= M_6 + M_7,\\
        C_{22} &= M_2 - M_3 + M_5 - M_7.
    \end{aligned}
\end{flalign}

Алгоритм Штрассена рекурсивно производит разбиение исходных матриц в соответствии с (\ref{eqn:mat-split}), вычисление результирующей матрицы согласно (\ref{eqn:aux-mat-strassen}), (\ref{eqn:c-mat-strassen}).

\section*{Вывод}

В данном разделе были рассмотрены алгоритмы умножения матриц: классический, алгоритм Винограда, алгоритм Штрассена.
Для алгоритма Винограда отдельно были рассмотрены возможные оптимизации, применимые при реализации.