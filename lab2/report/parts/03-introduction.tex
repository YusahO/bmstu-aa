\chapter*{Введение}
\addcontentsline{toc}{chapter}{Введение}

Умножение матриц является основным инструментом линейной алгебры и имеет многочисленные применения в математике, физике, программировании \cite{winograd-haskell}.

Целью данной лабораторной работы является изучение, реализация и исследование алгоритмов умножения матриц.

Необходимо выполнить следующие задачи:
\begin{enumerate}[]
    \item изучить следующие алгоритмы умножения матриц:
    \begin{itemize}
        \item классический алгоритм;
        \item алгоритм Винограда;
        \item оптимизированный алгоритм Винограда;
        \item влгоритм Штрассена;
    \end{itemize}
    \item реализовать данные алгоритмы;
    \item выполненить сравнительный анализ алгоритмов по затрачиваемым ресурсам (времени, памяти);
    \item описать и обосновать полученные результаты в отчете.
\end{enumerate}