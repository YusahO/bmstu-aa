\chapter{Технологическая часть}

В данном разделе приведены средства реализации программного обеспечения, сведения о модулях программы, листинг кода и функциональные тесты.

\section{Средства реализации}

В качестве языка программирования, используемого при написании данной лабораторной работы, был выбран C++ \cite{cpp-lang}, так как в нем имеется контейнер \texttt{std::wstring}, представляющий собой массив символов \texttt{std::wchar\_t}, и библиотека \texttt{<ctime>} \cite{ctime}, позволяющая производить замеры процессорного времени.

В качестве средства написания кода была выбрана кроссплатформенная среда разработки \textit{Visual Studio Code} за счет того, что она предоставляет функционал для проектирования, разработки и отладки ПО.

\section{Сведения о модулях программы}

Данная программа разбита на следующие модули:

\begin{itemize}
    \item \texttt{main.cpp}~--- файл, содержащий точку входа в программу;
    \item \texttt{matrix.cpp}~--- файл, содержащий функции создания матрицы, ее освобождения и вывода на экран;
    \item \texttt{algorithms.cpp}~--- файл, содержащий реализации алгоритмов поиска расстояний Левенштейна и Дамерау~---~Левенштейна;
    \item \texttt{measure.cpp}~--- файл, содержащий функции, замеряющие процессорное время выполнения реализуемых алгоритмов.
\end{itemize}

\clearpage
\section{Реализация алгоритмов}

\begin{lstlisting}[caption={Функция \texttt{min}, используемая в реализациях алгоритмов}]
template <typename T>
static T min(T x, T y, T z)
{
    return std::min(x, std::min(y, z));
}
\end{lstlisting}

\begin{lstlisting}[caption=Матричный алгоритм поиска расстояния Левенштейна (часть 1)]
int LevNonRec(const std::wstring &word1, const std::wstring &word2, bool verbose)
{
    int len1 = word1.length();
    int len2 = word2.length();

    int **mat = Matrix::Allocate(len2 + 1, len1 + 1);
    if (!mat)
        return -1;

    mat[0][0] = 0;
    for (int i = 0; i <= len2; ++i)
    {
        for (int j = 0; j <= len1; ++j)
        {
            if (i == 0)
            {
                mat[i][j] = j;
            }
            else if (j == 0)
            {
                mat[i][j] = i;
            }
\end{lstlisting}

\clearpage
\begin{lstlisting}[caption=Матричный алгоритм поиска расстояния Левенштейна (часть 2)]
            else
            {
                int cost = (word1[j - 1] == word2[i - 1]) ? 0 : 1;

                mat[i][j] = min(
                    mat[i - 1][j] + 1,
                    mat[i][j - 1] + 1,
                    mat[i - 1][j - 1] + cost);
            }
        }
    }

    if (verbose)
        Matrix::Print(mat, word1, word2);

    int res = mat[len2][len1];
    Matrix::Free(mat, len2 + 1);

    return res;
}
\end{lstlisting}

\begin{lstlisting}[caption=Матричный алгоритм поиска расстояния Дамерау~---~Левенштейна (часть 1)]
int DamLevNonRec(const std::wstring &word1, const std::wstring &word2, bool verbose)
{
    int len1 = word1.length();
    int len2 = word2.length();

    int **mat = Matrix::Allocate(len2 + 1, len1 + 1);
    if (!mat)
        return -1;

    mat[0][0] = 0;
    for (int i = 0; i <= len2; ++i)
    {
\end{lstlisting}

\clearpage
\begin{lstlisting}[caption=Матричный алгоритм поиска расстояния Дамерау~---~Левенштейна (часть 2)]
        for (int j = 0; j <= len1; ++j)
        {
            if (i == 0)
            {
                mat[i][j] = j;
            }
            else if (j == 0)
            {
                mat[i][j] = i;
            }
            else
            {
                int cost = (word1[j - 1] == word2[i - 1]) ? 0 : 1;

                mat[i][j] = min(
                    mat[i - 1][j] + 1,
                    mat[i][j - 1] + 1,
                    mat[i - 1][j - 1] + cost);

                if (word1[j - 2] == word2[i - 1] && word1[j - 1] == word2[i - 2])
                {
                    mat[i][j] = std::min(
                        mat[i][j],
                        mat[i - 2][j - 2] + 1);
                }
            }
        }
    }

    if (verbose)
        Matrix::Print(mat, word1, word2);

    int res = mat[len2][len1];
    Matrix::Free(mat, len2 + 1);

    return res;
}
\end{lstlisting}

\begin{lstlisting}[caption=Рекурсивный алгоритм поиска расстояния Дамерау~---~Левенштейна]
int DamLevRec(const std::wstring &word1, const std::wstring &word2, int ind1, int ind2)
{
    if (ind1 == 0)
        return ind2;
    if (ind2 == 0)
        return ind1;

    int cost = (word1[j - 1] == word2[i - 1]) ? 0 : 1;

    int res = min(
        DamLevRec(word1, word2, ind1, ind2 - 1) + 1,
        DamLevRec(word1, word2, ind1 - 1, ind2) + 1,
        DamLevRec(word1, word2, ind1 - 1, ind2 - 1) + cost);

    if (ind1 > 1 && ind2 > 1 && word1[ind1 - 1] == word2[ind2 - 2] && word1[ind1 - 2] == word2[ind2 - 1])
        res = std::min(res, DamLevRec(word1, word2, ind1 - 2, ind2 - 2) + 1);

    return res;
}
\end{lstlisting}

\begin{lstlisting}[caption=Рекурсивный алгоритм поиска расстояния Дамерау~---~Левенштейна с кэшированием (реализация) (часть 1)]
static int DamLevRecCacheImpl(const std::wstring &word1, const std::wstring &word2, int ind1, int ind2, int **memo)
{
    if (ind1 == 0)
        return ind2;

    if (ind2 == 0)
        return ind1;

    if (memo[ind2][ind1] != -1)
        return memo[ind2][ind1];
    
    int cost = (word1[j - 1] == word2[i - 1]) ? 0 : 1;
\end{lstlisting}

\begin{lstlisting}[caption=Рекурсивный алгоритм поиска расстояния Дамерау~---~Левенштейна с кэшированием (реализация) (часть 2)]
    int res = min(
        DamLevRecCacheImpl(word1, word2, ind1, ind2 - 1, memo) + 1,
        DamLevRecCacheImpl(word1, word2, ind1 - 1, ind2, memo) + 1,
        DamLevRecCacheImpl(word1, word2, ind1 - 1, ind2 - 1, memo) + cost
    );

    if (ind1 > 1 && ind2 > 1 && word1[ind1 - 1] == word2[ind2 - 2] && word1[ind1 - 2] == word2[ind2 - 1])
        res = std::min(res, DamLevRecCacheImpl(word1, word2, ind1 - 2, ind2 - 2, memo) + 1);

    memo[ind2][ind1] = res;
    return res;
}
\end{lstlisting}

\begin{lstlisting}[caption=Рекурсивный алгоритм поиска расстояния Дамерау~---~Левенштейна с кэшированием (оберточная функция)]
int DamLevRecCache(const std::wstring &word1, const std::wstring &word2)
{
    int len1 = word1.length();
    int len2 = word2.length();

    int **memo = Matrix::Allocate(len2 + 1, len1 + 1, -1);
    if (!memo)
        return -1;

    int res = DamLevRecCacheImpl(word1, word2, len1, len2, memo);

    Matrix::Free(memo, len2 + 1);
    return res;
}
\end{lstlisting}

\section{Функциональные тесты}

\begin{filecontents*}{parts/func_tests.csv}
а,б,1,1,1,1
a,b,1,1,1,1
r,r,0,0,0,0
сердце,солнце,3,3,3,3
стол,стул,1,1,1,1
кот,ток,2,2,2,2
кто,кот,2,1,1,1
Вениамин,Венгрия,4,4,4,4
стол,столы,1,1,1,1
\end{filecontents*}

\begin{table}[H]
    \small
	\begin{center}
		\caption{Функциональные тесты}
		\label{tbl:func_tests}
        \begin{tabular}{|c|c|c|c|c|c|}
            \hline
            \multicolumn{2}{|c|}{\bfseries Входные данные}
            & \multicolumn{4}{c|}{\bfseries Расстояние и алгоритм} \\ 
            \hline 
            &
            & \multicolumn{1}{c|}{\bfseries Левенштейна} 
            & \multicolumn{3}{c|}{\bfseries Дамерау~---~Левенштейна} \\ \cline{3-6}
            
            \bfseries Строка 1 & \bfseries Строка 2 & \bfseries Итеративный & \bfseries Итеративный
            
            & \multicolumn{2}{c|}{\bfseries Рекурсивный} \\ \cline{5-6}
            & & & & \bfseries Без кэша & \bfseries С кэшем

            \csvreader[separator=comma]{parts/func_tests.csv}{}{% 
                \\ \hline \csvcoli & \csvcolii & \csvcoliii & \csvcoliv & \csvcolv & \csvcolvi
            }
            \\ \hline
        \end{tabular}
	\end{center}
\end{table}

\section*{Вывод}
\addcontentsline{toc}{section}{Вывод}

Были реализованы алгоритмы Левенштейна (итеративно) и Дамерау~---~Левенштейна (итеративно, рекурсивно, рекурсивно с кэшированием).
Проведено тестирование реализованных алгортимов.
