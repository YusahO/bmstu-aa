\chapter{Технологическая часть}

В данном разделе приведены средства реализации программного обеспечения, сведения о модулях программы, листинг кода и функциональные тесты.

\section{Требования к ПО}

К программе предъявлен ряд требований:

\begin{itemize}
    \item на вход программе подаются две матрицы, каждая записана в отдельном текстовом файле;
    \item результатом умножения является матрица, выводимая на экран;
    \item программа должна позволять производить измерения процессорного времени, затрачиваемого на выполнение реализуемых алгоритмов.
\end{itemize}

\section{Средства реализации}

В качестве языка программирования, используемого при написании данной лабораторной работы, был выбран C++ \cite{cpp-lang}, так как в нем имеется контейнер \texttt{std::vector}, представляющий собой массив динамический массив данных произвольного типа, и библиотека \texttt{<ctime>} \cite{ctime}, позволяющая производить замеры процессорного времени.

В качестве средства написания кода была выбрана кроссплатформенная среда разработки \textit{Visual Studio Code} за счет того, что она предоставляет функционал для проектирования, разработки и отладки ПО.

\section{Сведения о модулях программы}

Данная программа разбита на следующие модули:

\begin{itemize}
    \item \texttt{main.cpp}~--- файл, содержащий точку входа в программу;
    \item \texttt{matrix.cpp}~--- файл, содержащий класс \texttt{Matrix}, реализующий необходимые для работы с матрицами функции;
    \item \texttt{algorithms.cpp}~--- файл, содержащий реализации алгоритмов умножения матриц;
    \item \texttt{measure.cpp}~--- файл, содержащий функции, замеряющие процессорное время выполнения реализуемых алгоритмов.
\end{itemize}

\clearpage
\section{Реализация алгоритмов}

\begin{lstlisting}[label={lst:common}, caption={Реализация классического алгоритма умножения матриц}]
Matrix Common(const Matrix &m1, const Matrix &m2)
{
    size_t rows1 = m1.rows();
    size_t cols1 = m1.columns();
    size_t cols2 = m2.columns();

    Matrix res(rows1, cols2, 0);

    for (size_t i = 0; i < rows1; ++i)
        for (size_t j = 0; j < cols2; ++j)
            for (size_t k = 0; k < cols1; ++k)
                res[i][j] = res[i][j] + m1[i][k] * m2[k][j];

    return res;
}
\end{lstlisting}

\begin{lstlisting}[label={lst:winograd}, caption={Реализация алгоритма Винограда}]
Matrix Winograd(const Matrix &m1, const Matrix &m2)
{
    size_t rows1 = m1.rows();
    size_t cols1 = m1.columns();
    size_t cols2 = m2.columns();

    Matrix res(rows1, rows1);

    std::vector<int> row_factors(rows1, 0);
    std::vector<int> col_factors(cols2, 0);

    for (size_t i = 0; i < rows1; ++i)
        for (size_t j = 0; j < cols1 / 2; ++j)
            row_factors[i] = row_factors[i] + m1[i][2 * j] * m1[i][2 * j + 1];

    for (size_t i = 0; i < cols2; ++i)
        for (size_t j = 0; j < cols1 / 2; ++j)
            col_factors[i] = col_factors[i] + m2[2 * j][i] * m2[2 * j + 1][i];

    for (size_t i = 0; i < rows1; ++i)
    {
        for (size_t j = 0; j < cols2; ++j)
        {
            res[i][j] = -row_factors[i] - col_factors[j];
            for (size_t k = 0; k < cols1 / 2; ++k)
            {
                res[i][j] = res[i][j] + (m1[i][2 * k] + m2[2 * k + 1][j]) *
                            (m1[i][2 * k + 1] + m2[2 * k][j]);
            }
        }
    }

    if (cols1 % 2)
    {
        for (size_t i = 0; i < rows1; ++i)
            for (size_t j = 0; j < cols2; ++j)
                res[i][j] = res[i][j] + m1[i][cols1 - 1] *
                            m2[cols1 - 1][j];
    }

    return res;
}
\end{lstlisting}

\begin{lstlisting}[label={lst:winograd-opt}, caption={Реализация алгоритма Винограда с оптимизациями}]
Matrix WinogradOpt(const Matrix &m1, const Matrix &m2)
{
    size_t rows1 = m1.rows();
    size_t cols1 = m1.columns();
    size_t cols2 = m2.columns();

    Matrix res(rows1, rows1);

    std::vector<int> row_factors(rows1, 0);
    std::vector<int> col_factors(cols2, 0);

    size_t half_cols1 = cols1 / 2;

    for (size_t i = 0; i < rows1; ++i)
        for (size_t j = 0; j < half_cols1; ++j)
            row_factors[i] += m1[i][j << 1] * m1[i][(j << 1) + 1];

    for (size_t i = 0; i < cols2; ++i)
        for (size_t j = 0; j < half_cols1; ++j)
            col_factors[i] += m2[j << 1][i] * m2[(j << 1) + 1][i];

    for (size_t i = 0; i < rows1; ++i)
    {
        for (size_t j = 0; j < cols2; ++j)
        {
            res[i][j] = -row_factors[i] - col_factors[j];
            for (size_t k = 0; k < half_cols1; ++k)
            {
                size_t k_shifted = k << 1;
                res[i][j] += (m1[i][k_shifted] + m2[k_shifted + 1][j]) *
                    (m1[i][k_shifted + 1] + m2[k_shifted][j]);
            }
        }
    }

    if (cols1 % 2)
    {
        for (size_t i = 0; i < rows1; ++i)
            for (size_t j = 0; j < cols2; ++j)
                res[i][j] += m1[i][cols1 - 1] *
                m2[cols1 - 1][j];
    }

    return res;
}
\end{lstlisting}

\begin{lstlisting}[label={lst:strassen}, caption={Реализация алгоритма Штрассена}]
Matrix Strassen(const Matrix &m1, const Matrix &m2)
{
    size_t n = m1.rows() / 2;
    size_t rows = m1.rows();

    if (rows <= 2)
    {
        return Common(m1, m2);
    }

    auto a11 = m1.slice(0, n, 0, n);
    auto a12 = m1.slice(0, n, n, rows);
    auto a21 = m1.slice(n, rows, 0, n);
    auto a22 = m1.slice(n, rows, n, rows);

    auto b11 = m2.slice(0, n, 0, n);
    auto b12 = m2.slice(0, n, n, rows);
    auto b21 = m2.slice(n, rows, 0, n);
    auto b22 = m2.slice(n, rows, n, rows);

    auto p1 = Strassen(a11 + a22, b11 + b22);
    auto p2 = Strassen(a22, b21 - b11);
    auto p3 = Strassen(a11 + a12, b22);
    auto p4 = Strassen(a12 - a22, b21 + b22);
    auto p5 = Strassen(a11, b12 - b22);
    auto p6 = Strassen(a21 + a22, b11);
    auto p7 = Strassen(a11 - a21, b11 + b12);

    auto c11 = p1 + p2 - p3 + p4;
    auto c12 = p5 + p3;
    auto c21 = p6 + p2;
    auto c22 = p5 + p1 - p6 - p7;

    return Matrix::combine(c11, c12, c21, c22);
}
\end{lstlisting}

\section{Функциональные тесты}

На таблице \ref{tbl:func-tests-classic} представлены функциональные тесты стандартного алгоритма умножения матриц.

На таблице \ref{tbl:func-tests-winograd} представлены функциональные тесты алгоритма Винограда.

На таблице \ref{tbl:func-tests-strassen} показаны функциональные тесты алгоритма Штрассена.


\begin{table}[H]
    \small
	\begin{center}
		\caption{Функциональные тесты для классического алгоритма}
		\label{tbl:func-tests-classic}  
            \begin{tabular}{|c|c|c|c|}
            \hline
            \textbf{Матрица 1} & \textbf{Матрица 2} & \textbf{Ожидаемый результат} & \textbf{Фактический результат} \\
            \hline
            $\begin{pmatrix}
                9 
            \end{pmatrix}$ &
            $\begin{pmatrix}
                3
            \end{pmatrix}$ &
            $\begin{pmatrix}
                27
            \end{pmatrix}$ &
            $\begin{pmatrix}
                27
            \end{pmatrix}$ \\
            \hline
            $\begin{pmatrix}
                1 \\ 2 \\ 3
            \end{pmatrix}$ &
            $\begin{pmatrix}
                4 & 5 & 6
            \end{pmatrix}$ &
            $\begin{pmatrix}
                4 & 5 & 6 \\
                8 & 10 & 12 \\
                12 & 15 & 18
            \end{pmatrix}$ &
            $\begin{pmatrix}
                4 & 5 & 6 \\
                8 & 10 & 12 \\
                12 & 15 & 18
            \end{pmatrix}$ \\
            \hline
            $\begin{pmatrix}
                2 & 10 \\
                3 & 7
            \end{pmatrix}$ &
            $\begin{pmatrix}
                1 & 0 \\
                0 & 1
            \end{pmatrix}$ &
            $\begin{pmatrix}
                2 & 10 \\
                3 & 7
            \end{pmatrix}$ &
            $\begin{pmatrix}
                2 & 10 \\
                3 & 7
            \end{pmatrix}$ \\
            \hline
            $\begin{pmatrix}
                4 & 2 & 9 \\
                1 & 3 & 1 \\
                0 & 5 & 3
            \end{pmatrix}$ &
            $\begin{pmatrix}
                2 & 0 & 0 \\
                0 & 2 & 0 \\
                0 & 0 & 1                
            \end{pmatrix}$ &
            $\begin{pmatrix}
                8 & 4 & 9 \\
                2 & 6 & 1 \\
                0 & 10 & 3
            \end{pmatrix}$ &
            $\begin{pmatrix}
                8 & 4 & 9 \\
                2 & 6 & 1 \\
                0 & 10 & 3
            \end{pmatrix}$ \\
            \hline
            $\begin{pmatrix}
                4 & 2 & 9 \\
                1 & 3 & 1 \\
                0 & 5 & 3
            \end{pmatrix}$ &
            $\begin{pmatrix}
                &
            \end{pmatrix}$ &
            \textit{Сообщение об ошибке} &
            \textit{Сообщение об ошибке} \\
            \hline
            $\begin{pmatrix}
                1 & 2 & 3
            \end{pmatrix}$ &
            $\begin{pmatrix}
                4 & 5 & 6
            \end{pmatrix}$ &
            \textit{Сообщение об ошибке} &
            \textit{Сообщение об ошибке} \\
            \hline
        \end{tabular}
	\end{center}
\end{table}


\begin{table}[H]
    \small
	\begin{center}
		\caption{Функциональные тесты для алгоритма Винограда}
		\label{tbl:func-tests-winograd}  
            \begin{tabular}{|c|c|c|c|}
            \hline
            \textbf{Матрица 1} & \textbf{Матрица 2} & \textbf{Ожидаемый результат} & \textbf{Фактический результат} \\
            \hline
            $\begin{pmatrix}
                9 
            \end{pmatrix}$ &
            $\begin{pmatrix}
                3
            \end{pmatrix}$ &
            $\begin{pmatrix}
                27
            \end{pmatrix}$ &
            $\begin{pmatrix}
                27
            \end{pmatrix}$ \\
            \hline
            $\begin{pmatrix}
                1 \\ 2 \\ 3
            \end{pmatrix}$ &
            $\begin{pmatrix}
                4 & 5 & 6
            \end{pmatrix}$ &
            \textit{Сообщение об ошибке} &
            \textit{Сообщение об ошибке} \\
            \hline
            $\begin{pmatrix}
                2 & 10 \\
                3 & 7
            \end{pmatrix}$ &
            $\begin{pmatrix}
                1 & 0 \\
                0 & 1
            \end{pmatrix}$ &
            $\begin{pmatrix}
                2 & 10 \\
                3 & 7
            \end{pmatrix}$ &
            $\begin{pmatrix}
                2 & 10 \\
                3 & 7
            \end{pmatrix}$ \\
            \hline
            $\begin{pmatrix}
                4 & 2 & 9 \\
                1 & 3 & 1 \\
                0 & 5 & 3
            \end{pmatrix}$ &
            $\begin{pmatrix}
                2 & 0 & 0 \\
                0 & 2 & 0 \\
                0 & 0 & 1                
            \end{pmatrix}$ &
            $\begin{pmatrix}
                8 & 4 & 9 \\
                2 & 6 & 1 \\
                0 & 10 & 3
            \end{pmatrix}$ &
            $\begin{pmatrix}
                8 & 4 & 9 \\
                2 & 6 & 1 \\
                0 & 10 & 3
            \end{pmatrix}$ \\
            \hline
            $\begin{pmatrix}
                4 & 2 & 9 \\
                1 & 3 & 1 \\
                0 & 5 & 3
            \end{pmatrix}$ &
            $\begin{pmatrix}
                &
            \end{pmatrix}$ &
            \textit{Сообщение об ошибке} &
            \textit{Сообщение об ошибке} \\
            \hline
            $\begin{pmatrix}
                1 & 2 & 3
            \end{pmatrix}$ &
            $\begin{pmatrix}
                4 & 5 & 6
            \end{pmatrix}$ &
            \textit{Сообщение об ошибке} &
            \textit{Сообщение об ошибке} \\
            \hline
        \end{tabular}
	\end{center}
\end{table}

\begin{table}[H]
    \small
	\begin{center}
		\caption{Функциональные тесты для алгоритма Штрассена}
		\label{tbl:func-tests-strassen}  
            \begin{tabular}{|c|c|c|c|}
            \hline
            \textbf{Матрица 1} & \textbf{Матрица 2} & \textbf{Ожидаемый результат} & \textbf{Фактический результат} \\
            \hline
            $\begin{pmatrix}
                9 
            \end{pmatrix}$ &
            $\begin{pmatrix}
                3
            \end{pmatrix}$ &
            $\begin{pmatrix}
                27
            \end{pmatrix}$ &
            $\begin{pmatrix}
                27
            \end{pmatrix}$ \\
            \hline
            $\begin{pmatrix}
                1 \\ 2 \\ 3
            \end{pmatrix}$ &
            $\begin{pmatrix}
                4 & 5 & 6
            \end{pmatrix}$ &
            \textit{Сообщение об ошибке} &
            \textit{Сообщение об ошибке} \\
            \hline
            $\begin{pmatrix}
                2 & 10 \\
                3 & 7
            \end{pmatrix}$ &
            $\begin{pmatrix}
                1 & 0 \\
                0 & 1
            \end{pmatrix}$ &
            $\begin{pmatrix}
                2 & 10 \\
                3 & 7
            \end{pmatrix}$ &
            $\begin{pmatrix}
                2 & 10 \\
                3 & 7
            \end{pmatrix}$ \\
            \hline
            $\begin{pmatrix}
                4 & 2 & 9 \\
                1 & 3 & 1 \\
                0 & 5 & 3
            \end{pmatrix}$ &
            $\begin{pmatrix}
                2 & 0 & 0 \\
                0 & 2 & 0 \\
                0 & 0 & 1                
            \end{pmatrix}$ &
            \textit{Сообщение об ошибке} &
            \textit{Сообщение об ошибке} \\
            \hline
            $\begin{pmatrix}
                4 & 2 & 9 \\
                1 & 3 & 1 \\
                0 & 5 & 3
            \end{pmatrix}$ &
            $\begin{pmatrix}
                &
            \end{pmatrix}$ &
            \textit{Сообщение об ошибке} &
            \textit{Сообщение об ошибке} \\
            \hline
            $\begin{pmatrix}
                1 & 2 & 3
            \end{pmatrix}$ &
            $\begin{pmatrix}
                4 & 5 & 6
            \end{pmatrix}$ &
            \textit{Сообщение об ошибке} &
            \textit{Сообщение об ошибке} \\
            \hline
        \end{tabular}
	\end{center}
\end{table}

\section*{Вывод}
\addcontentsline{toc}{section}{Вывод}

Были реализованы алгоритмы умножения матриц (классический, алгоритм Винограда, алгоритм Штрассена).
Проведено тестирование реализованных алгортимов.
