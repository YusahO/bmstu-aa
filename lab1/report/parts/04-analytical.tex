\chapter{Аналитическая часть}

\section{Расстояние Левенштейна}

Расстояние Левенштейна между двумя строками~--- это минимальное количество операций вставки одного символа, удаления одного символа и замены одного символа на другой, необходимых для превращения строки в другую \cite{analysis-lev-damlev}.

Введем следующие обозначения операций:
\begin{itemize}
    \item $w(a, b)$ -- цена замены символа $a$ на символ $b$;
    \item $w(\varepsilon, b)$ – цена вставки символа $b$; 
    \item $w(a, \varepsilon)$ – цена удаления символа $a$.
\end{itemize}

Каждая операция имеет определенную цену:
\begin{itemize}
    \item \textbf{M} (от англ. match): $w(a, a) = 0$;
    \item \textbf{R} (от англ. replace): $w(a, b) = 1, a \ne b$;
    \item \textbf{I} (от англ. insert): $w(\varepsilon, b) = 1$;
    \item \textbf{D} (от англ. delete): $w(a, \varepsilon) = 1$.
\end{itemize}

Пусть имеется две строки $S_1$ и $S_2$ длиной $m$ и $n$ соотвественно.
Расстояние Левенштейна $d(S_1, S_2) = D(m, n)$ рассчитывается по следующей рекуррентной формуле \cite{prog-impl-lev}:

\begin{equation}
    \label{eqn:recur-lev}
    D(m, n) =
    \begin{cases}
        0, &\text{i = 0, j = 0}\\
        i, &\text{j = 0, i > 0}\\
        j, &\text{i = 0, j > 0}\\
        \min
        \begin{cases}
            D(i, j - 1) + 1,\\
            D(i - 1, j) + 1,\\
            D(i - 1, j - 1) + \text{m}(S_1[i], S_2[j]),
        \end{cases} &\text{j > 0, i > 0}
    \end{cases}
\end{equation}

\noindent где сравнение символов строк $S_1$ и $S_2$ производится следующим образом:

\begin{equation}
    \text{m}(a, b) = 
    \begin{cases}
        0, &\text{если $a = b$}\\
        1, &\text{иначе}
    \end{cases}
\end{equation}

\subsection{Нерекурсивный алгоритм нахождения расстояния Левенштейна}
При увеличении значений $m$, $n$ алгоритм поиска расстояния Левенштейна, использующий рекурсию, становится малоэффективным по времени за счет того, что в ходе работы алгоритма промежуточные значения $D(i, j)$ вычисляются неоднократно.

Результаты промежуточных вычислений можно сохранять в матрицу размером $(n + 1) \times (m + 1)$, где $m$~--- длина строки $S_1$, $n$~--- длина строки $S_2$.

В ячейке $[i, j]$ матрицы хранится значение $D(S_1[1..i], S_2[1..j])$.
Первому элементу матрицы присвоено значение $0$.
Вся матрица заполняется в соотвествии с соотношением (\ref{eqn:recur-lev}).

\section{\texorpdfstring{Расстояние Дамерау~---~Левенштейна}{}}

Расстояние Дамерау~---~Левенштейна~--- это расширение расстояния Левенштейна, определяющееся как минимальное количество операций вставки, удаления, замены и транспозиции T (от англ. transposition).

Расстояние Дамерау~---~Левенштейна задается следующей рекуррентной формулой:
\begin{multline}
    D(m, n) =\\ =
    \begin{cases}
        0, &\text{i = 0, j = 0}\\
        i, &\text{j = 0, i > 0}\\
        j, &\text{i = 0, j > 0}\\
        \min
        \begin{cases}
            D(i, j - 1) + 1,\\
            D(i - 1, j) + 1,\\
            D(i - 1, j - 1),\\
            D(i - 2, j - 2) + 1,
        \end{cases} 
        &\begin{aligned}
            & \text{если $i, j > 1$}, \\
            & S_{1}[i] = S_{2}[j - 1], \\
            & S_{1}[i - 1] = S_{2}[j],
        \end{aligned} \\
        \min
        \begin{cases}
            D(i - 1, j) + 1, \\
            D(i, j - 1) + 1, \\
            D(i - 1, j - 1) + \text{m}(S_1[i], S_2[j])
        \end{cases} &\text{иначе.}
    \end{cases}
    \label{eqn:recur-damlev}
\end{multline}

\subsection{\texorpdfstring{Рекурсивный алгоритм нахождения расстояния Дамерау~---~Левенштейна}{}}

Рекурсивный алгоритм поиска расстояния Дамерау~---~Левенштейна реализует формулу (\ref{eqn:recur-damlev}) следующим образом:
\begin{enumerate}
    \item Если одна из строк пустая, возвращается длина другой строки.
    \item Если последние символы двух строк совпадают, рекурсивно вызывается функция для остатков строк (без последних символов).
    \item Иначе рекурсивно вызываются четыре варианта преобразования строки:
    \begin{itemize}
        \item \textbf{Вставка}: к результату рекурсивного вызова для остатка первой строки добавляется 1.
        \item \textbf{Удаление}: к результату рекурсивного вызова для остатка второй строки добавляется 1.
        \item \textbf{Замена}: к результату рекурсивного вызова для остатков строк добавляется 1.
        \item \textbf{Транспозиция}: если последние и предпоследние символы двух строк совпадают, к результату рекурсивного вызова для остатка строк добавляется 1.
    \end{itemize}
    \item Возвращается минимальное из четырех вариантов значение.
\end{enumerate}


\subsection{\texorpdfstring{Рекурсивный алгоритм нахождения расстояния Дамерау~---~Левенштейна с кэшированием}{}}

При увеличении $m$ и $n$ рекурсивная реализация алгоритма поиска расстояния Дамерау~---~Левенштейна становится крайне не эффективной по времени, так как промежуточные значения расстояний между подстроками вычисляются неоднократно.
Избавиться от повторяющихся вычислений можно с помощью матрицы $A_{m, n}$, в которую по ходу работы алгоритма сохраняются соотвествующие промежуточные значения $D(i, j)$ расстояний.

Размер матрицы-кэша равен $(n + 1) \times (m + 1)$.

\subsection{\texorpdfstring{Нерекурсивный алгоритм нахождения расстояния Дамерау~---~Левенштейна}{}}

При увеличении значений $m$, $n$ алгоритм нахождения расстояния Дамерау~---~Левенштейна, использующий рекурсию, становится менее эффективным по времени, поэтому вместо рекурсивной реализации можно использовать итеративную, для хранения промежуточных значений $D(i, j)$ применяющую матрицу размером $(n + 1) \times (m + 1)$.

В ячейке $[i, j]$ матрицы хранится значение $D(S_1[1..i], S_2[1..j])$.
Первому элементу матрицы присвоено значение $0$.
Вся матрица заполняется в соотвествии с соотношением (\ref{eqn:recur-damlev}).

\section*{Вывод}
\addcontentsline{toc}{section}{Вывод}

В данном разделе были рассмотрены алгоритмы нахождения расстояний Левенштейна и Дамерау~---~Левенштейна~--- их рекурсивные и итеративные реализации.
Также была рассмотрена оптимизация алгоритма поиска расстояния Дамерау~---~Левенштейна с помощью кэширования.