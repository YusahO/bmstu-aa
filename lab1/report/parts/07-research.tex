\chapter{Исследовательская часть}

\section{Технические характеристики}

Технические характеристики устройства, на котором выполнялись замеры по времени:

\begin{itemize}
    \item Процессор: Intel i5-1035G1 (8) @ 3.600GHz.
    \item Оперативная память: 16 ГБайт.
    \item Операционная система: Manjaro Linux x86\_64 (версия ядра Linux 5.15.131-1-MANJARO).
\end{itemize}

Во время проведения измерений времени ноутбук был подключен к сети электропитания и был нагружен только системными приложениями.

\section{Демонстрация работы программы}

На рисунке \ref{fig:prog-demo} показан пример работы разработанной программы для случая, когда пользователь выбирает действие <<Запуск алгоритмов поиска расстояния Левенштейна>> и вводит строки <<кошка>> и <<броненосец>>.

\clearpage
\begin{figure}[h]
    \centering
    \includegraphics[height=0.6\textheight]{images/prog_demo.png}
    \caption{Демонстрация работы программы}
    \label{fig:prog-demo}
\end{figure}

\section{Временные характеристики}

Исследование временных характеристик реализованных алгоритмов производилось на строках длинами:

\begin{itemize}
    \item 1 -- 10 c шагом 1 для всех реализаций;
    \item 10 -- 200 c шагом 10 только для нерекурсивных реализаций;
\end{itemize}

В силу того, что время работы алгоритмов может колебаться в связи с различными процессами, происходящими в системе, для обеспечения более точных результатов измерения для каждого алгоритма повторялись 500 раз, а затем бралось их среднее арифметическое значение.

На рисунке \ref{fig:nonrec-time} показаны зависимости времени выполнения матричных реализаций алгоритмов Левенштейна и Дамерау\,--\,Левенштейна от длин входящих строк.

\begin{figure}[H]
    \centering
    \includesvg[width=1.0\textwidth]{images/nonrec.svg}
    \caption{Результат измерений времени работы нерекурсивных реализаций алгоритмов поиска расстояний Левенштейна и Дамерау\,--\,Левенштейна}
    \label{fig:nonrec-time}
\end{figure}

\begin{figure}[H]
    \centering
    \includesvg[width=1.0\textwidth]{images/rec.svg}
    \caption{Результат измерений времени работы рекурсивных реализаций алгоритма поиска расстояния Дамерау\,--\,Левенштейна}
    \label{fig:rec-time}
\end{figure}

\begin{figure}[H]
    \centering
    \includesvg[width=1.0\textwidth]{images/dl_all.svg}
    \caption{Результат измерений времени работы реализаций алгоритмов поиска расстояния Дамерау\,--\,Левенштейна}
    \label{fig:dl-all-time}
\end{figure}

\section{Характеристики по памяти}

Введем следующие обозначения:

\begin{itemize}
    \item $m$~--- длина строки $S_1$;
    \item $n$~--- длина строки $S_2$;
    \item $\text{size}(v)$~--- функция, вычисляющая размер входного параметра $v$ в байтах;
    \item $char$~--- тип данных, используемый для хранения символа строки;
    \item $int$~--- целочисленный тип данных.
\end{itemize}

Теоретически оценим объем используемой памяти итеративной реа\-лизацией алгоритма поиска расстояния Левенштейна:

\begin{multline}
    M_{LevIter} = (m + 1) \cdot (n + 1) \cdot \text{size}(int) + (m + n) \cdot \text{size}(char) + \\
    + \text{size}(int**) + (m + 1) \cdot \text{size}(int*) + \\
    + 3 \cdot \text{size}(int) + 2 \cdot \text{size}(int),
\end{multline}

\noindent где:
\begin{itemize}
    \item $(m + 1) \cdot (n + 1) \cdot \text{size}(int)$~--- размер матрицы;
    \item $\text{size}(int**)$~--- размер указателя на матрицу;
    \item $(m + 1) \cdot \text{size}(int*)$~--- размер указателей на строки матрицы;
    \item $(m + n) \cdot \text{size}(char)$~--- размер двух входных строк;
    \item $2 \cdot \text{size}(int)$~--- размер переменных, хранящих длину строк;
    \item $3 \cdot \text{size}(int)$~--- размер дополнительных переменных.
\end{itemize}

Для алгоритма поиска расстояния Дамерау\,--\,Левенштейна теорети\-ческая оценка объема используемой памяти идентична.

Произведем оценку затрат по памяти для рекурсивных реализаций алгоритма нахождения расстояния Дамерау\,--\,Левенштейна.

Сперва рассчитаем объем памяти, используемой каждым вызовом функции поиска расстояния Дамерау\,--\,Левенштейна:
\begin{equation}
    M_{call} = (m + n) \cdot \text{size}(char) + 2 \cdot \text{size}(int) + 3 \cdot \text{size}(int) + 8
\end{equation}

\noindent где
\begin{itemize}
    \item $(m + n) \cdot \text{size}(char)$~--- объем памяти, используемый для хранения двух строк;
    \item $2 \cdot \text{size}(int)$~--- размер двух входных строк;
    \item $3 \cdot \text{size}(int)$~--- размер дополнительных переменных;
    \item 8 байт~--- адрес возврата.
\end{itemize}

Максимальная глубина стека вызовов при рекурсивной реализации равна сумме длин входящих строк, поэтому максимальный расход памяти равен

\begin{equation}
    M_{DLRec} = (m + n) \cdot M_{call}
\end{equation}

\noindent где

\begin{itemize}
    \item $m + n$~--- максимальная глубина стека вызовов;
    \item $M_{call}$~--- затраты по памяти для одного рекурсивного вызова;
\end{itemize}

Рекурсивная реализация алгоритма поиска расстояния Дамерау\,--\,Левенштейна с кэшированием для хранения промежуточных значений использует матрицу (кэш), размер которой можно рассчитать следующим образом:

\begin{multline}
	M_{cash} = (n + 1) \cdot (m + 1) \cdot \text{size}(int) +\\+ \text{size}(int **) + (m + 1) \cdot \text{size}(int *)
\end{multline}
где:
\begin{itemize}
    \item $(n + 1) \cdot (m + 1)$~--- количество элементов в кэше;
	\item $\text{size}(int **)$~--- размер указателя на матрицу;
	\item $(m + 1) \cdot \text{size}(int *)$~--- размер указателя на строки матрицы.
\end{itemize}

Таким образом, затраты по памяти для рекурсивного алгоритма нахождения расстояния Дамерау\,--\,Левенштейна с использованием кэша:

\begin{equation}
    M_{DLRecCache} = M_{DLRec} + M_{cache}
\end{equation}

\section{Вывод}

В результате исследования выяснилось, что алгоритм 
