\chapter*{Заключение}
\addcontentsline{toc}{chapter}{Заключение}

В результате выполнения лаботарторной работы по исследованию алгоритмов поиска расстояния Левенштейна и Дамерау~---~Левенштейна были решены следующие задачи:
\begin{enumerate}
    \item Описаны алгоритмы поиска расстояний Левенштейна и Дамерау~---~Левенштейна;
    \item Разработаны и реализованы соответствующие алгоритмы;
    \item Создан программный продукт, позволяющий протестировать реализованные алгоритмы;
    \item Проведен сравнительный анализ процессорного времени выполнения реализованных алгоритмов.
    \begin{itemize}
        \item При небольших длинах строк (длина < 5 симв.) разница между временем выполнения нерекурсивных реализаций алгоритмов Левенштейна и Дамерау~---~Левенштейна незначительна.
        При увеличении длин строк время работы матричного алгоритма поиска расстояния Дамерау~---~Левенштейна становится больше, в связи с обработкой условия о перестановке символов.
        \item Рекурсивный алгоритм поиска расстояния Дамерау~---~Левенштейна выполняется на порядок дольше, чем тот же алгоритм, использующий кэширование.
        \item Время работы матричного и рекурсивного с кэшированием алгоритмов поиска расстояния Дамерау~---~Левенштейна приблизительно равно.
    \end{itemize}
\item Выполнена теоретическая оценка объема затрачиваемой памяти каждым из реализованных алгоритмов: нерекурсивные алгоритмы и рекурсивный алгоритм с кэшированием, требуют больше памяти по сравнению с рекурсивным, не использующим кэширование, так как максимальный размер использования памяти у матричных реализаций увеличивается пропорционально произведений длин входящих строк, а у рекурсивных~--- пропорционально их сумме.
\end{enumerate}
