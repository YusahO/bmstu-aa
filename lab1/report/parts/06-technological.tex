\chapter{Технологическая часть}

В данном разделе приведены средства реализации программного обеспечения, сведения о модулях программы, листинг кода и функцио\-нальные тесты.

\section{Средства реализации}

В качестве языка, используемого при написании данной лабора\-торной работы, был выбран язык C++ \cite{cpp-lang}. Этот выбор обусловлен тем, что в данном языке программирования имеется контей\-нер \texttt{std::wstring}, представляющий собой массив символов типа \texttt{std::wchar\_t}. Также в языке С++ имеется библиотека \texttt{<ctime>}, позволяющая выполнять замеры процессорного времени.

В качестве средства написания кода была выбрана кроссплатфор\-менная среда разработки \textit{Visual Studio Code}, т.к. она предоставляет ши\-рокий функционал для проектирования, разработки и отладки ПО.

\section{Сведения о модулях программы}

Данная программа разбита на следующие модули:

\begin{itemize}
    \item \texttt{main.cpp} --- файл, содержащий точку входа в программу;
    \item \texttt{matrix.cpp} --- файл, содержащий функции создания матрицы, ее освобождения и вывода на экран;
    \item \texttt{algorithms.cpp} --- файл, содержащий реализации алгоритмов поиска расстояний Левенштейна и Дамерау-Левенштейна;
    \item \texttt{measure.cpp} --- файл, содержащий функции, замеряющие процес\-сорное время выполнения реализуемых алгоритмов.
\end{itemize}

\section{Реализация алгоритмов}

\section{Функциональные тесты}

