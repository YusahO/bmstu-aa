\chapter{Конструкторская часть}

В данном разделе будут приведены схемы алгоритмов нахождения расстояний Левенштейна и Дамерау-Левенштейна, приведены описание используемых типов данных и структуры программного обеспечения.

\section{Требования к программному обеспе\-чению}

К программе предъявлен ряд функциональных требований:

\begin{itemize}
    \item наличие интерфейса для выбора действий;
    \item возможность ввода строк;
    \item возможность обработки строк, состоящих как из латинских символов, так и из кириллических;
    \item возможность произвести замеры процессорного времени работы ре\-ализованных алгоритмов поиска расстояний Левенштейна и Дамерау-Левенштейна.
\end{itemize}

\section{Требования вводу}

\begin{enumerate}
    \item На вход реализованным алгоритмам подаются две строки.
    \item Строки могут включать как латинские, так и кириллические сим\-волы.
    \item Буквы нижнего и верхнего регистра считаются разными символами.
\end{enumerate}

\section{Разработка алгоритмов}

На рисунке \ref{fig:lev-iter} представлена схема матричного алгоритма поиска расстояния Левенштейна.
На рисунках \ref{fig:dam-lev-iter} приведены схемы матричной, рекурсивной и рекурсивной с кэшированием реализаций алгоритма нахож\-дения расстояния Дамерау-Левенштейна.

\begin{figure}[H]
    \centering
    \includesvg[width=1.0\textwidth]{images/lev-iter.svg}
    \caption{Схема матричного алгоритма Левенштейна}
    \label{fig:lev-iter}
\end{figure}

\begin{figure}[H]
    \centering
    \includesvg[width=1.0\textwidth]{images/dam-lev-iter.svg}
    \caption{Схема матричного алгоритма Дамерау-Левенштейна}
    \label{fig:dam-lev-iter}
\end{figure}


\section{Описание используемых типов данных}

При реализации алгоритмов будут использованы следующие типы данных:

\begin{itemize}
    \item \textit{строка} --- массив символов типа \texttt{wchar\_t};
    \item \textit{длина строки} --- значение длины строки типа \texttt{int};
    \item \textit{матрица} --- двумерный массив значений типа \texttt{int}.
\end{itemize}

\section*{Вывод}
\addcontentsline{toc}{section}{Вывод}

Были реализованы алгоритмы Левенштейна (итеративно) и Дамерау-Левенштейна (итеративно, рекурсивно, рекурсивно с кэшированием). Проведено тестирование реализованных алгортимов.
