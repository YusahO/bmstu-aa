\chapter*{Введение}
\addcontentsline{toc}{chapter}{Введение}

\textbf{Расстояние Левенштейна} (также называемое редакционным рас\-стоянием
или дистанцией редактирования) --- это метрика, которая измеря\-ет разницу
между двумя строками. Определяет минимальное количество опе\-раций 
вставки, удаления и замены символов, необходимых для преобра\-зования
одной строки в другую.

\textbf{Расстояние Дамерау-Левенштейна} является расширением рас\-стояния 
Левенштейна, которое включает дополнительную операцию --- транспозицию,
чтобы обработать случаи, когда символы меняются местами или
переу\-порядочиваются.

Расстояния Левенштейна и Дамерау-Левенштейна используются
при решении следующих задач:
\begin{enumerate}
    \item корректировка поискового запроса;
    \item классификация текстов;
    \item распознавание речи;
    \item определение сходства между текстами;
\end{enumerate}

\textbf{Целью} данной лабораторной работы является изучение, 
реализация и исследование алгоритмов поиска расстояний Левенштейна 
и Дамерау-Левенштейна.

Необходимо выполнить следующие \textbf{задачи}:
\begin{enumerate}[]
    \item изучить алгоритмы Левенштейна и Дамерау-Левенштейна для 
    нахождения редакционного расстояния между строками;
    \item реализовать данные алгоритмы;
    \item выполненить сравнительный анализ алгоритмов по затрачиваемым ресурсам (времени, памяти);
    \item описать и обосновать полученные результаты в отчете.
\end{enumerate}