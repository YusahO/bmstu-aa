\chapter{Аналитическая часть}

В данном разделе приведена информация, касающаяся основ конвейерной обработки данных.

\section{Конвейерная обработка данных}

Конвейер~--- организация вычислений, при которой увеличивается количество выполняемых инструкций за единицу времени за счет использования принципов параллельности.

Конвейеризация в компьютерной обработке данных основана на разбиении выполнения функций на более мелкие этапы, называемые ступенями, и выделении отдельной аппаратуры для каждой из них.
Это позволяет организовать передачу данных от одного этапа к следующему, что увеличивает производительность за счет одновременного выполнения нескольких команд на различных ступенях конвейера.

Хотя конвейеризация не уменьшает время выполнения отдельной команды, она повышает пропускную способность процессора, что приводит к более быстрому выполнению программы по сравнению с простой, не конвейерной схемой.

\section{Нечеткий алгоритм кластеризации c-средних}

Кластерный анализ~--- это ряд математических методов интеллектуального анализа данных, предназначенных для разбиения множества исследуемых объектов на компактные группы, называемые кластерами. Под объектами кластерного анализа подразумеваются предметы исследования, нуждающиеся в кластеризации по некоторым признакам.
Признаки объектов могут иметь как непрерывные, так и дискретные значения~\cite{c-means}.

Метод c-средних~--- итеративный нечеткий алгоритм кластеризации.
В данном методе кластеры являются нечеткими множествами, и каждый объект из выборки исходных данных относится одновременно ко всем кластерам с различной степенью принадлежности.
Таким образом, матрица принадлежности объектов к кластерам (или матрица разбиения) содержит не бинарные, а вещественные значения, принадлежащие отрезку $[0; 1]$~\cite{c-means}.

Пусть $X$~--- исходный набор данных размера $N$. Обновление матрицы принадлежности и списка центров кластеров производится в 5 этапов:
\begin{enumerate}
	\item инициализация матрицы центров кластеров $W$ случайными значениями;
	\item инициализация матрицы разбиения $U = (\mu_{ij})$ следующим образом:
	\begin{equation}
		\label{eqn:u-calc}
		\begin{gathered}
			\mu_{ij}^{(t)} = \frac{1}{\mathlarger{\sum_{l=1}^C}\left( \frac{d_{ij}}{d_{il}} \right)^\frac{2}{m-1} }, i = 1, \dots, N; j,l = 1, \dots C \\
			\mu_{ij} = 
			\begin{cases}
				1, & \text{если } d_{ij} = 0 \\
				0, & \text{для } l \ne j,
			\end{cases}
		\end{gathered}
	\end{equation}
	где $t$~--- номер итерации,
	\\ $C$~--- количество кластеров,
	\\ $m$~--- показатель нечеткости, регулирующий точность разбиения,
	\\ $d_{ij}$~--- расстояние от $x_i$ до $w_j$, $d_{ij} = \|x_i - w_{j}^{(t)}\|$;
	\item увеличить $t$ на $1$ и рассчитать матрицу $W^{(t)}$ по формуле~(\ref{eqn:w-calc})
	\begin{equation}
		\label{eqn:w-calc}
		W_j^{(t)} = \frac{\mathlarger{\sum_{i=1}^N}\left( \mu_{ij}^{(t-1)} \right)^m x_i}{\mathlarger{\sum_{i=1}^N}\left( \mu_{ij}^{(t-1)} \right)^m}, j = 1, \dots, C;
	\end{equation}
	\item вычислить матрицу разбиения $U^{(t)}$ согласно соотношению~(\ref{eqn:u-calc});
	\item если $\| U^{(t)} < U^{(t-1)} \| \ge \varepsilon$ перейти на шаг 3~\cite{c-means-steps}.
\end{enumerate}

\section{Описание алгоритмов}

В качестве операций, выполняющихся на конвейере, взяты следующие:
\begin{enumerate}
	\item обработка файла, описывающего датасет файлов MIDI~\cite{midi};
	\item извлечение биграмм нот;
	\item кластеризация нечетким алгоритмом c-средних.
\end{enumerate}

Ленты конвейера (обработчики) будут передавать друг другу заявки.
Первый этап, или обработчик, будет формировать заявку, которая будет передаваться от этапа к этапу.
Заявка содержит:
\begin{enumerate}
	\item массив нот, извлеченных из файла MIDI;
	\item массив биграмм нот;
	\item результат кластеризации: матрица разбиения и список центроидов кластеров.
\end{enumerate}

\section*{Вывод}
В данном разделе было рассмотрено понятие конвейерной обработки, а также выбраны этапы для обработки датасета файлов MIDI.
Также был рассмотрен и описан алгоритм кластеризации c-средних.

