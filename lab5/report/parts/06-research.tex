\chapter{Исследовательская часть}

В данном разделе приведены технические характеристики устройства, на котором проводилось измерение времени работы программного обеспечения, а также результаты замеров времени.

\section{Технические характеристики}

Технические характеристики устройства, на котором выполнялись замеры по времени:

\begin{itemize}
	\item процессор: AMD Ryzen 7 5800X @ 3.800 ГГц, 8 физ. ядер, 16 лог. ядер;
	\item оперативная память: 32 ГБайт.
	\item операционная система: Manjaro Linux x86\_64 (версия ядра Linux~6.5.12-1-MANJARO).
\end{itemize}

Измерения проводились на стационарном компьютере.
Во время проведения измерений устройство было нагружено только системными приложениями.

\section{Демонстрация работы программы}

На рисунке \ref{img:prog_demo} продемонстрирована работа программы для случая, когда пользователь выбрал пункт 2 <<Запустить последовательную обработку датасета файлов MIDI>>, выбрал файл с описанием датасета под номером 1 и ввел следующие значения:
\begin{itemize}
	\item количество заявок~--- $10$,
	\item число кластеров~--- $3$,
	\item значение показателя нечеткости~--- $2$,
	\item значение порога сходимости~--- $1$,
	\item максимальное к-во итераций~--- $10$.
\end{itemize}

\includeimage
	{prog_demo}
	{f}
	{H}
	{1\textwidth}
	{Демонстрация работы программы}

\section{Временные характеристики}

Исследование временных характеристик реализуемых алгоритмов производилось 2 раза:
\begin{enumerate}
	\item 
\end{enumerate}
Наборы данных генерировались из равномерного распределения.

\section{Вывод}

В результате исследования реализуемых алгоритмов по времени выполнения можно сделать следующие выводы:
\begin{enumerate}
	\item 
\end{enumerate}
