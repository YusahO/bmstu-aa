\chapter*{ЗАКЛЮЧЕНИЕ}
\addcontentsline{toc}{chapter}{ЗАКЛЮЧЕНИЕ}

В результате выполнения лабораторной работы по организации асинхронного взаимодействия потоков вычисления на примере конвейерных вычислений решены все поставленные задачи:
\begin{enumerate}
    \item описана организация конвейерной обработки данных;
	\item описаны алгоритмы обработки данных, которые использованы в текущей лабораторной работе;
	\item определены средства программной реализации;
    \item реализована программа, выполняющая конвейерную обработку с количеством лент не менее трех в однопоточной и многопоточной реализациях;
	\item проведен сравнительный анализ процессорного времени выполнения реализованных алгоритмов:
	\begin{itemize}
	\item при варьировании числа заявок линейный алгоритм обработки датасета, состоящем из двух файлов MIDI, выполнялся дольше конвейерного в среднем в $214.5$ раз;
	\item при варьировании числа файлов линейный алгоритм обработки датасета при 10 заявках выполнялся дольше конвейерного в среднем в $103.7$ раз.
	\end{itemize}
\end{enumerate}