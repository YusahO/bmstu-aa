\chapter{Конструкторская часть}

\section{Требования к программному обеспечению}

К программе предъявлен ряд требований:
\begin{enumerate}
	\item наличие интерфейса для вывода результатов работы кластеризации;
	\item наличие возможности ввода массива входных точек $(x, y)$ из текстового файла;
\end{enumerate}

\section{Разработка алгоритмов}

На рисунке \ref{img:main_thread} показана схема функции главного потока нечеткого алгоритма c-средних.

На рисунке \ref{img:calc_membership} представлена схема функции заполнения матрицы принадлежности.

На рисунке \ref{img:recalc_cc} показана схема функции сдвига центров кластеров.

На рисунке \ref{img:calc_dist} показана схема функции расчета изменения позиции центров кластеров.

\includeimage
	{main_thread}
	{f}
	{H}
	{0.7\textwidth}
	{Функция главного потока нечеткого алгоритма c-средних}
	
\includeimage
	{calc_membership}
	{f}
	{H}
	{1\textwidth}
	{Функции заполнения матрицы принадлежности}
	
\includeimage
	{recalc_cc}
	{f}
	{H}
	{1\textwidth}
	{Функции сдвига центров кластеров}
	
\includeimage
	{calc_dist}
	{f}
	{H}
	{1\textwidth}
	{Функции расчета изменения позиции центров кластеров}

\section*{Вывод}
\addcontentsline{toc}{section}{Вывод}

В данном разделе выдвинуты требования к программному обеспечению и представлена схема нечеткого алгоритма c-средних.