\chapter{Аналитическая часть}

Кластерный анализ~--- это ряд математических методов интеллектуального анализа данных, предназначенных для разбиения множества исследуемых объектов на компактные группы, называемые кластерами. Под объектами кластерного анализа подразумеваются предметы исследования, нуждающиеся в кластеризации по некоторым признакам.
Признаки объектов могут иметь как непрерывные, так и дискретные значения~\cite{c-means}.

Метод c-средних~--- итеративный нечеткий алгоритм кластеризации.
В данном методе кластеры являются нечеткими множествами, и каждый объект из выборки исходных данных относится одновременно ко всем кластерам с различной степенью принадлежности.
Таким образом, матрица принадлежности объектов к кластерам содержит не бинарные, а вещественные значения, принадлежащие отрезку $[0; 1]$~\cite{c-means}.

Метод c-средних предполагает заполнение матрицы разбиения (принадлежности) $U = \{u_{ij}\}$ следующим образом~\cite{c-means}:
\begin{equation}
	u_{ij} = \frac{1}{\sum_{k=1}^{p} \left( \frac{d^2 (x_j, c_i) }{ d^2 (x_j, c_k) } \right)^\frac{1}{w - 1}},
\end{equation}
где $d$~--- расстояние между объектом и центром кластера,
\\ $C = \{c_i\}_{i=1}^p$~--- множество центров кластеров,
\\ $X = \{x_j\}_{j=1}^n$~--- множество объектов,
\\ $w$~--- показатель нечеткости, регулирующий точность разбиения.

\section*{Вывод}
В данном разделе было рассмотрено понятие кластеризации.
Также был описан нечеткий алгоритм c-средних.