\chapter{Технологическая часть}

\section{Средства реализации}

Для реализация нечеткого алгоритма c-средних был выбран язык \textit{C++}~\cite{cpp-lang} так как данный язык был использован в лабораторной работе №4.

\section{Реализация алгоритмов}

На листинге \ref{lst:c_means_main.cpp} показана реализация основной функции нечеткого алгоритма c-средних.

На листинге \ref{lst:calc_membership.cpp} представлена реализация функции рассчета элементов матрицы принадлежности.

На листинге \ref{lst:recalc_cc.cpp} показана реализация функции перерасчета положений центров кластеров.

На листинге \ref{lst:calc_dist.cpp} представлена реализация функции рассчета смещений положений кластеров относительно их старой позиции.

\includelistingpretty
	{c_means_main.cpp}
	{c++}
	{Реализация основной функции нечеткого алгоритма c-средних}
	
\includelistingpretty
	{calc_membership.cpp}
	{c++}
	{Реализация функции рассчета элементов матрицы принадлежности}
	
\includelistingpretty
	{recalc_cc.cpp}
	{c++}
	{Реализация функции перерасчета положений центров кластеров}
	
\includelistingpretty
	{calc_dist.cpp}
	{c++}
	{Реализация функции рассчета смещений положений кластеров относительно их старой позиции}

\section*{Вывод}
\addcontentsline{toc}{section}{Вывод}
В данном разделе были приведены сведения о средствах реализации нечеткого алгоритма c-средних.
Также была предоставлена реализация алгоритма.