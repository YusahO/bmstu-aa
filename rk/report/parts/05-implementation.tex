\chapter{Технологическая часть}

\section{Средства реализации}

Для реализация нечеткого алгоритма c-средних был выбран язык \textit{C++}~\cite{cpp-lang}, так как данный язык был использован в лабораторной работе №4.

\section{Реализация алгоритмов}

На листинге \ref{lst:c_means_main.cpp} показана реализация алгоритма главного потока нечеткого алгоритма c-средних.

\includelistingpretty
	{c_means_main.cpp}
	{c++}
	{Реализация алгоритма главного потока нечеткого алгоритма c-средних}

На листинге \ref{lst:calc_membership.cpp} представлена реализация алгоритма расчета элементов матрицы принадлежности.
	
\includelistingpretty
	{calc_membership.cpp}
	{c++}
	{Реализация алгоритма расчета элементов матрицы принадлежности}

На листинге \ref{lst:recalc_cc.cpp} показана реализация алгоритма перерасчета положений центров кластеров.

\includelistingpretty
	{recalc_cc.cpp}
	{c++}
	{Реализация алгоритма перерасчета положений центров кластеров}

На листинге \ref{lst:calc_dist.cpp} представлена реализация алгоритма расчета смещений положений кластеров относительно их старой позиции.

\includelistingpretty
	{calc_dist.cpp}
	{c++}
	{Реализация алгоритма расчета смещений положений кластеров относительно их старой позиции}

\section*{Вывод}

В данном разделе были приведены сведения о средствах реализации нечеткого алгоритма c-средних.
Также была предоставлена реализация разработанных алгоритмов.