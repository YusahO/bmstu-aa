\chapter{Конструкторская часть}

В данном разделе разработаны схемы реализаций нечеткого алгоритма кластеризации c-средних.

\section{Требования к программному обеспечению}

К программному обеспечению предъявлен ряд требований:
\begin{enumerate}
	\item наличие интерфейса для выбора действий;
	\item возможность загрузки массива исходных данных, записанных в текстовый файл;
	\item работа с массивами и <<нативными>> потоками.
\end{enumerate}

\section{Разработка алгоритмов}

На рисунке~\ref{img:scheme_consequent} представлена схема последовательного нечеткого алгоритма кластеризации c-средних.

\includeimage
	{scheme_consequent}
	{f}
	{H}
	{0.95\textwidth}
	{Схема последовательного нечеткого алгоритма c-средних}

На рисунке~\ref{img:scheme_parallel} представлена схема алгоритма главного потока, вызывающего вспомогательные потоки.

\includeimage
	{scheme_parallel}
	{f}
	{H}
	{0.85\textwidth}
	{Схема алгоритма главного потока, вызывающего вспомогательные потоки}
	
На рисунке~\ref{img:scheme_calc_mship} представлена схема алгоритма, производящего подготовку к вызову вспомогательных потоков и ожидание их завершение.

\includeimage
	{scheme_calc_mship}
	{f}
	{H}
	{0.6\textwidth}
	{Схема алгоритма, производящего подготовку к вызову вспомогательных потоков и ожидание их завершение}

На рисунке~\ref{img:scheme_worker} представлена схема алгоритма, выполняющегося внутри вспомогательных потоков для обновления элементов матрицы принадлежности.

\includeimage
	{scheme_worker}
	{f}
	{H}
	{0.6\textwidth}
	{Схема алгоритма, выполняющегося внутри вспомогательных потоков для обновления элементов матрицы принадлежности}

\section*{Вывод}
В данном разделе были перечислены требования к программному обеспечению и построены схемы рассматриваемых алгоритмов.