\chapter*{ЗАКЛЮЧЕНИЕ}
\addcontentsline{toc}{chapter}{ЗАКЛЮЧЕНИЕ}

В результате выполнения лаботаторной работы по исследованию алгоритмов сортировок решены следующие задачи:
\begin{enumerate}
	\item описан нечеткий алгоритм кластеризации c-средних;
	\item разработана параллельная версия алгоритма;
	\item определены средства программной реализации;
	\item реализованы последовательная и параллельная версии алгоритма;
	\item проведен сравнительный анализ процессорного времени выполнения реализованных алгоритмов:
	\begin{itemize}
		\item при варьировании размера датасета последовательная версия нечеткого алгоритма кластеризации c-средних выполнялась в среднем в $92.6$ раз дольше, чем параллельная; 
		\item при варьировании числа потоков последовательная версия нечеткого алгоритма кластеризации c-средних выполнялась в среднем $72.7$ раз дольше, чем параллельная;
		\item наименьшее время работы многопоточной реализации алгоритма достигается при 8 вспомогательных потоках; наибольшее же время выполнения алгоритма достигается при 4 потоках;
	\end{itemize}
	таким образом, рекомендуется использование 8 вспомогательных потоков, т.~к. при таком количестве временные затраты на создание потоков, переключение аппаратного контекста и синхронизацию ниже, чем получаемая скорость обработки набора данных.
\end{enumerate}