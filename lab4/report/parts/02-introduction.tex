\chapter*{ВВЕДЕНИЕ}
\addcontentsline{toc}{chapter}{ВВЕДЕНИЕ}

Кластеризация данных является важным инструментом в области машинного обучения.
Она позволяет группировать данные на основе их сходства и отделить их от остальных~\cite{inro-clust}. 

Целью данной лабораторной работы является получение навыков организации параллельного выполнения операций.

Для достижения поставленной цели необходимо решить следующие задачи:
\begin{enumerate}[label={\arabic*)}]
    \item описать нечеткий алгоритм кластеризации c-средних;
    \item разработать параллельную версию алгоритма;
    \item определить средства программной реализации;
    \item реализовать данные алгоритмы;
    \item выполнить замеры процессорного времени работы различных реализаций алгоритма и произвести анализ полученных данных.
\end{enumerate}