\chapter{Исследовательская часть}

В данном разделе приведены технические характеристики устройства, на котором проводилось измерение времени работы программного обеспечения, а также результаты замеров времени.

\section{Технические характеристики}

Технические характеристики устройства, на котором выполнялись замеры по времени:

\begin{itemize}
	\item процессор: AMD Ryzen 7 5800X @ 3.800 ГГц, 8 физ. ядер, 16 лог. ядер;
	\item оперативная память: 32 ГБайт.
	\item операционная система: Manjaro Linux x86\_64 (версия ядра Linux~6.5.12-1-MANJARO).
\end{itemize}

Измерения проводились на стационарном компьютере.
Во время проведения измерений устройство было нагружено только системными приложениями.

\section{Демонстрация работы программы}

На рисунке \ref{img:prog_demo} продемонстрирована работа программы для случая, когда пользователь выбрал пункт 1 <<Кластеризация методом c-средних>> и ввел следующие значения:
\begin{itemize}
	\item число кластеров~--- $2$,
	\item значение показателя нечеткости~--- $3$,
	\item значение порога сходимости~--- $0.3$,
	\item максимально к-во итераций~--- $10$,
	\item число дополнительных потоков~--- $16$.
\end{itemize}

\includeimage
	{prog_demo}
	{f}
	{H}
	{1\textwidth}
	{Демонстрация работы программы}

\section{Временные характеристики}

Исследование временных характеристик реализуемых алгоритмов производилось 2 раза:
\begin{enumerate}
	\item 
\end{enumerate}

\section{Вывод}
