\chapter{Исследовательская часть}

В данном разделе приведены технические характеристики устройства, на котором проводилось измерение времени работы программного обеспечения, а также результаты замеров времени.

\section{Технические характеристики}

Технические характеристики устройства, на котором выполнялись замеры по времени:

\begin{itemize}
	\item процессор: AMD Ryzen 7 5800X @ 3.800 ГГц, 8 физ. ядер, 16 лог. ядер;
	\item оперативная память: 32 ГБайт.
	\item операционная система: Manjaro Linux x86\_64 (версия ядра Linux~6.5.12-1-MANJARO).
\end{itemize}

Измерения проводились на стационарном компьютере.
Во время проведения измерений устройство было нагружено только системными приложениями.

\section{Демонстрация работы программы}

На рисунке \ref{img:prog_demo} продемонстрирована работа программы для случая, когда пользователь выбрал пункт 1 <<Кластеризация методом c-средних>> и ввел следующие значения:
\begin{itemize}
	\item число кластеров~--- $2$,
	\item значение показателя нечеткости~--- $3$,
	\item значение порога сходимости~--- $0.3$,
	\item максимально к-во итераций~--- $10$,
	\item число дополнительных потоков~--- $16$.
\end{itemize}

\includeimage
	{prog_demo}
	{f}
	{H}
	{1\textwidth}
	{Демонстрация работы программы}

\section{Временные характеристики}

Исследование временных характеристик реализуемых алгоритмов производилось 2 раза:
\begin{enumerate}
	\item при изменении числа потоков $1, 2, 4, \dots, 128$ для набора данных из $10000$ точек;
	\item при изменении размера набора данных от $5000$ до $30000$ с шагом $5000$ и при использовании 1 вспомогательного потока.
\end{enumerate}
Наборы данных генерировались из равномерного распределения.

На рисунке~\ref{img:time_threads} представлены результаты измерения времени работы реализуемых алгоритмов при варьировании числа потоков; таблица~\ref{tbl:time_threads} содержит значения, по которой был построен данный график.

\includeimage
	{time_threads}
	{f}
	{H}
	{1\textwidth}
	{Сравнение последовательной и параллельной реализаций нечеткого алгоритма кластеризации c-средних при изменении числа потоков}
	
\begin{filecontents*}{parts/threads.csv}
	1,10000,1064832.83,16832.37
	2,10000,1064832.83,18643.05
	4,10000,1064832.83,18717.83
	8,10000,1064832.83,11900.51
	16,10000,1064832.83,12271.80
	32,10000,1064832.83,13015.43
	64,10000,1064832.83,13797.37
	128,10000,1064832.83,15447.34
\end{filecontents*}

\begin{table}[H]
	\centering
	\caption{Результаты измерения времени работы реализуемых алгоритмов при варьировании числа потоков}
	\label{tbl:time_threads}
	\begin{tabular}{|c|c|S|S|}
		\hline
		~ & ~ & \multicolumn{2}{c|}{Время (мс)} \\
		\cline{3-4}
		К-во потоков (шт.) & Размер (элем.) & Послед{.} & Парал{.} \\ \hline
		\csvreader[
			separator=comma,
			late after line = \\\hline
		]{parts/threads.csv}{}{% 
			\csvcoli & \csvcolii & \csvcoliii & \csvcoliv 
		}
	\end{tabular}
\end{table}

На рисунке~\ref{img:time_threads_paral} показаны результаты измерения времени работы параллельного нечеткого алгоритма кластеризации c-средних в зависимости от числа потоков.

\includeimage
	{time_threads_paral}
	{f}
	{H}
	{1\textwidth}
	{График времени работы параллельного нечеткого алгоритма кластеризации c-средних}
	
На рисунке~\ref{img:time_lengths} представлены результаты измерения времени работы последовательной и параллельной версий рассматриваемого алгоритма в зависимости от размера набора данных; в таблице~\ref{tbl:time_lengths} приведены значения, по которым строился данный график.

\includeimage
	{time_lengths}
	{f}
	{H}
	{1\textwidth}
	{Сравнение последовательной и параллельной реализаций нечеткого алгоритма кластеризации c-средних при изменении размера датасета}

\begin{filecontents*}{parts/lengths.csv}
	1,5000,527403.44,5823.20
	1,10000,1054345.96,11429.15
	1,15000,1581842.77,17065.27
	1,20000,2107652.32,22645.99
	1,25000,2634434.62,28160.84
	1,30000,3161812.22,33805.39
\end{filecontents*}

\begin{table}[H]
	\centering
	\caption{Результаты измерения времени работы реализуемых алгоритмов при варьировании размера датасета}
	\label{tbl:time_lengths}
	\begin{tabular}{|c|c|S|S|}
		\hline
		~ & ~ & \multicolumn{2}{c|}{Время (мс)} \\
		\cline{3-4}
		К-во потоков (шт.) & Размер (элем.) & Послед{.} & Парал{.} \\ \hline
		\csvreader[
			separator=comma,
			late after line = \\\hline
		]{parts/lengths.csv}{}{% 
			\csvcoli & \csvcolii & \csvcoliii & \csvcoliv 
		}
	\end{tabular}
\end{table}

\section{Вывод}

В результате исследования реализуемых алгоритмов по времени выполнения можно сделать следующие выводы:
\begin{enumerate}
	\item при варьировании размера датасета последовательная версия нечеткого алгоритма кластеризации c-средних выполнялась в среднем в $92.6$ раз дольше, чем параллельная; 
	\item при варьировании числа потоков последовательная версия нечеткого алгоритма кластеризации c-средних выполнялась в среднем $72.7$ раз дольше, чем параллельная;
	\item наименьшее время работы многопоточной реализации алгоритма достигается при 8 вспомогательных потоках; наибольшее же время выполнения алгоритма достигается при 4 потоках.
\end{enumerate}
Таким образом, рекомендуется использование 8 вспомогательных потоков, т.~к. при таком количестве временные затраты на создание потоков, переключение аппаратного контекста и синхронизацию ниже, чем получаемая скорость обработки набора данных.
