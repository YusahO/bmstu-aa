\chapter{Технологическая часть}

В данном разделе описаны средства реализации программного обеспечения, а также листинги и функциональные тесты.

\section{Средства реализации}

В качестве языка программирования, используемого при написании данной лабораторной работы, был выбран C++ \cite{cpp-lang}, так как в нем имеется контейнер \texttt{std::vector}, представляющий собой динамический массив данных произвольного типа, и библиотека \texttt{<ctime>} \cite{ctime}, позволяющая производить замеры процессорного времени.
% todo cite

Для создания потоков и работы с ними был использован класс \texttt{thread} из стандартной библиотеки C++.
Листинг \ref{lst:thread_example.cpp} содержит пример работы с описанным классом, каждый экземпляр которого представляет собой поток операционной системы, позволяющий нескольким функциям выполняться одновременно.

\includelistingpretty
	{thread_example.cpp}
	{c++}
	{Пример работы с классом thread}

Поток начинает свою работу сразу после создания объекта класса \texttt{thread}, запуская функцию, переданную в его конструктор, с переданными туда же параметрами.
Данном примере был запущен 1 поток, который выполнит функцию \texttt{func}, которая запишет число 30 в переменную \texttt{res}. 

\section{Сведения о модулях программы}

Данная программа разбита на следующие модули:

\begin{itemize}
	\item \texttt{main.cpp}~--- файл, содержащий точку входа в программу;
	\item \texttt{algorithms.cpp}~--- файл, содержащий последовательную и параллельную реализации нечеткого алгоритма c-средних;
	\item \texttt{measure.cpp}~--- файл, содержащий функции, замеряющие процессорное время выполнения реализуемых алгоритмов.
\end{itemize}

\section{Реализация алгоритмов}

На листинге~\ref{lst:c_means_cons.cpp} представлена реализация последовательной версии нечеткого алгоритма кластеризации c-средних.

\includelistingpretty
	{c_means_cons.cpp}
	{c++}
	{Реализация последовательного алгоритма кластеризации c-средних}
	
На листинге~\ref{lst:c_means_cons.cpp} представлена реализация алгоритма основного потока, запускающего вспомогательного потоки.

\includelistingpretty
	{c_means_main_thread.cpp}
	{c++}
	{Функция основного потока, запускающего вспомогательные потоки}
	
На листинге~\ref{lst:calc_membership.cpp} представлена реализация алгоритма функции, вызывающей вспомогательные потоки вычисления значений элементов матрицы принадлежности.
Для продолжения работы с матрицей принадлежности, содержащей обновленные значения, необходимо дождаться завершения работы всех потоков, выполняющих обновление элементов матрицы.

\includelistingpretty
	{calc_membership.cpp}
	{c++}
	{Функция, вызывающая вспомогательные потоки вычисления значений элементов матрицы принадлежности}

На листинге~\ref{lst:calc_membership_worker.cpp} представлена реализация алгоритма вычисления значений элементов матрицы принадлежности во вспомогательном потоке.

\includelistingpretty
	{calc_membership_worker.cpp}
	{c++}
	{Реализация алгоритма вычисления значений элементов матрицы принадлежности во вспомогательном потоке}

\section{Функциональные тесты}

В таблице~\ref{tbl:func-tests} представлены результаты функционального тестирования реализованных алгоритмов кластеризации для двух наборов точек:
\begin{enumerate}
	\item набор точек 1: $[(0, 0), (0, 2)]$,
	\item набор точек 2: $[(0, 0), (0, 2), (10, 40)]$.
\end{enumerate}
Многопоточная версия алгоритма тестировалась при числе потоков 10.
Также некоторые параметры алгоритмов были заданы заранее:
\begin{itemize}
	\item показатель нечеткости \texttt{m = 2},
	\item максимальное к-во итераций \texttt{max\_iters = 100},
	\item порог сходимости \texttt{conv\_threshold = 1}.
\end{itemize}
Все тесты пройдены успешно.

\begin{table}[H]
	\caption{Результаты функционального тестирования}
	\label{tbl:func-tests}
	\begin{tabularx}{\textwidth}{|X|c|c|c|}
		\hline 
		\multicolumn{2}{|c|}{Входные данные} & \multicolumn{2}{c|}{Результат} \\
		\hline
		Точки & К-во класт.  & Послед. & Парал. \\
		\hline
		Набор точек 1 & 1 
		& $ \begin{pmatrix}
			1 \\
			1
		\end{pmatrix} $
		& $ \begin{pmatrix}
			1 \\
			1
		\end{pmatrix} $
		\\ \hline
		Набор точек 1 & 2 
		& $ \begin{pmatrix}
			0.296 & 0.704 \\
			0.699 & 0.301 
		\end{pmatrix} $
		& $ \begin{pmatrix}
			0.296 & 0.704 \\
			0.699 & 0.301 
		\end{pmatrix} $ 
		\\ \hline
		Набор точек 2 & 3 
		& $ \begin{pmatrix}
			0.999 & 0.001 & 0.001 \\
			0.999 & 0.001 & 0.001 \\
			0.000 & 0.020 & 0.980
		\end{pmatrix} $
		& $ \begin{pmatrix}
			0.999 & 0.001 & 0.001 \\
			0.999 & 0.001 & 0.001 \\
			0.000 & 0.020 & 0.980
		\end{pmatrix} $
		\\ \hline
	\end{tabularx}
\end{table}

\section*{Вывод}
\addcontentsline{toc}{section}{Вывод}

В данном разделе были рассмотрены средства реализации, а также представлен листинг реализаций последовательного и параллельного нечеткого алгоритма кластеризации c-средних.